% Part: computability
% Chapter: recursive-functions
% Section: non-pr-functions

\documentclass[../../../include/open-logic-section]{subfiles}

\begin{document}

\olfileid{cmp}{rec}{npr}
\olsection{Non-Primitive Recursive Functions}

The primitive recursive functions do not exhaust the intuitively
computable functions. It should be intuitively clear that we can make
a list of all the unary primitive recursive functions,
$f_0$, $f_1$, $f_2$,~\dots such that we can effectively compute the value of
$f_x$ on input $y$; in other words, the function $g(x,y)$, defined by
\[
g(x,y) = f_x(y)
\]
is computable. But then so is the function
\begin{eqnarray*}
h(x) & = & g(x,x) + 1 \\
& = & f_x(x) +1.
\end{eqnarray*}
For each primitive recursive function $f_i$, the value of $h$ and
$f_i$ differ at $i$. So $h$ is computable, but not primitive
recursive; and one can say the same about $g$. This is an
``effective'' version of Cantor's diagonalization argument.

One can provide more explicit examples of computable functions that
are not primitive recursive. For example, let the notation $g^n(x)$
denote $g(g(\dots g(x)))$, with $n$ $g$'s in all; and define a
sequence $g_0,g_1,\dots$ of functions by
\begin{eqnarray*}
g_0(x) & = & x+1 \\
g_{n + 1}(x) & = & g_n^x(x)
\end{eqnarray*}
You can confirm that each function $g_n$ is primitive recursive. Each
successive function grows much faster than the one before; $g_1(x)$ is
equal to $2x$, $g_2(x)$ is equal to $2^x \cdot x$, and $g_3(x)$ grows
roughly like an exponential stack of $x$ $2$'s. Ackermann's function is
essentially the function $G(x) = g_x(x)$, and one can show that this
grows faster than any primitive recursive function.

Let us return to the issue of enumerating the primitive recursive
functions. Remember that we have assigned symbolic notations to each
primitive recursive function; so it suffices to enumerate
notations. We can assign a natural number $\#(F)$ to each notation $F$,
recursively, as follows:
\begin{eqnarray*}
\#(0) & = & \langle 0 \rangle \\
\#(S) & = & \langle 1 \rangle \\
\#(\Proj{n}{i}) & = & \langle 2, n, i \rangle \\
\#(\fn{Comp}_{k,l}[H,G_0,\dots,G_{k-1}]) & = & \langle
3,k,l,\#(H),\#(G_0),\dots,\#(G_{k-1}) \rangle \\
\#(\fn{Rec}_l[G,H]) & = & \langle 4, l, \#(G), \#(H) \rangle
\end{eqnarray*}
Here we are using the fact that every sequence of numbers can be viewed
as a natural number, using the codes from the last section. The upshot
is that every code is assigned a natural number. Of course, some
sequences (and hence some numbers) do not correspond to notations; but
we can let $f_i$ be the unary primitive recursive function with
notation coded as $i$, if $i$ codes such a notation; and the constant
$0$ function otherwise. The net result is that we have an explicit way of
enumerating the unary primitive recursive functions.

(In fact, some functions, like the constant zero function, will appear
more than once on the list. This is not just an artifact of our
coding, but also a result of the fact that the constant zero function has
more than one notation. We will later see that one can not computably
avoid these repetitions; for example, there is no computable function
that decides whether or not a given notation represents the constant
zero function.)

We can now take the function $g(x,y)$ to be given by $f_x(y)$, where
$f_x$ refers to the enumeration we have just described. How do we know
that $g(x,y)$ is computable? Intuitively, this is clear: to compute
$g(x,y)$, first ``unpack'' $x$, and see if it is a notation for a unary
function. If it is, compute the value of that function on input~$y$.

\begin{tagblock}{TMs}
\begin{digress}
You may already be convinced that (with some work!) one can write a
program (say, in Java or C++) that does this; and now we can appeal to
the Church-Turing thesis, which says that anything that, intuitively,
is computable can be computed by a Turing machine.

Of course, a more direct way to show that $g(x,y)$ is computable is to
describe a Turing machine that computes it, explicitly. This would,
in particular, avoid the Church-Turing thesis and appeals to
intuition. Soon we will have built up enough machinery to show
that $g(x,y)$ is computable, appealing to a model of computation that
can be \emph{simulated} on a Turing machine: namely, the recursive
functions.
\end{digress}
\end{tagblock}

\end{document}
