% Part: conditional-logics
% Chapter: minimal-change-semantics
% Section: true-false

\documentclass[../../../include/open-logic-section]{subfiles}

\begin{document}

\olfileid{con}{min}{tf}

\olsection{Truth and Falsity of Counterfactuals}

A counterfactual $!A \cif !B$ is (non-vacuously) true if the closest
$!A$-worlds are all $!B$-worlds, as depicted in \olref{fig:true}.
\begin{figure}
\begin{center}
\begin{tikzpicture}[scale=.7]\small
  \spheresystem{5}
  \draw (0,0) node {$w$};
  \propositionintersect{0}{4}{40}{3.5}
  {\tikzset{proposition/.append={smooth,tension=1.4}}
  \proposition[shift={(1.6,-1)}]{90}{1}{30}{4}}
  \path (1.5,3) node[below] {$\formula{B}$};
  \path (3,0) node {$\formula{A}$};
\end{tikzpicture}
\caption{Non-vacuously true counterfactual}
\ollabel{fig:true}
\end{center}
\end{figure}
A counterfactual is also true at $w$ if the system of spheres
around~$w$ has no $!A$-admitting spheres at all. In that case it is
\emph{vacuously} true (see  \olref{fig:vacuous}).
\begin{figure}
\begin{center}
\begin{tikzpicture}[scale=.7]\small
  \spheresystem{5}
  \draw (0,0) node {$w$};
  \proposition{0}{6.5}{40}{3.5}
  {\tikzset{proposition/.append={smooth,tension=1.4}}
  \proposition[shift={(1.2,-1)}]{90}{1}{30}{4}}
  \path (1.5,3) node[below] {$\formula{B}$};
  \spherepos{0}{7}{node {$\formula{A}$}}
\end{tikzpicture}
\caption{Vacuously true counterfactual}
\ollabel{fig:vacuous}
\end{center}
\end{figure}

It can be false in two ways. One way is if the closest $!A$-worlds are
not all $!B$-worlds, but some of them are. In this case, $!A \cif
\lnot !B$ is also false (see \olref{fig:false}).
\begin{figure}
\begin{center}
\begin{tikzpicture}[scale=.7]\small
  \spheresystem{5}
  \draw (0,0) node {$w$};
  \propositionintersect{0}{4}{40}{3.5}
  {\tikzset{proposition/.append={smooth,tension=1.4}}
  \proposition[shift={(1.6,0)}]{90}{1}{30}{3}}
  \path (1.5,3) node[below] {$\formula{B}$};
  \path (3,0) node {$\formula{A}$};
\end{tikzpicture}
\caption{False counterfactual, false opposite}
\ollabel{fig:false}
\end{center}
\end{figure}
If the closest $!A$-worlds do not overlap with the $!B$-worlds at all,
then $!A \cif !B$. But, in this case all the closest $!A$-worlds are
$\lnot !B$-worlds, and so $!A \cif \lnot !B$ is true (see
\olref{fig:false-opposite}).
\begin{figure}
\begin{center}
\begin{tikzpicture}[scale=.7]\small
  \spheresystem{5}
  \draw (0,0) node {$w$};
  \propositionintersect{0}{4}{40}{3.5}
  {\tikzset{proposition/.append={smooth,tension=1.4}}
  \proposition[shift={(-1,0)}]{90}{1}{30}{3}}
  \path (-1,3) node[below] {$\formula{B}$};
  \path (3,0) node {$\formula{A}$};
\end{tikzpicture}
\caption{False counterfactual, true opposite}
\ollabel{fig:false-opposite}
\end{center}
\end{figure}

In contrast to the strict conditional, counterfactuals may be
contingent. Consider the sphere model in \olref{fig:contingent}. The
$!A$-worlds closest to~$u$ are all $!B$-worlds, so $\mSat{M}{!A \cif
  !B}[u]$. But there are $!A$-worlds closest to~$v$ which are not
$!B$-worlds, so $\mSat/{M}{!A \cif !B}[v]$.
\begin{figure}
\begin{center}
\begin{tikzpicture}[scale=.6]\small
  \spheresystem{7}
  \spheresystem[shift={(0:3.1)},dashed]{4}
  \propositionintersect{45}{4}{40}{4.5}
  \begin{scope}
    \clip \propositionplot{45}{4}{40}{4.5} ;
    \spherelayer[shift={(0:3.1)}]{3}
  \end{scope}
  \proposition[shift={(1.4,-.5)}]{90}{1}{35}{5}
  \path (0,0) node {$u$};
  \path (0:3.1) node {$v$};
  \spherepos{35}{9}{node {$\formula{A}$}}
  \spherepos{80}{9}{node {$\formula{B}$}}
\end{tikzpicture}
\end{center}
\caption{Contingent counterfactual}
\ollabel{fig:contingent}
\end{figure}
\end{document}
