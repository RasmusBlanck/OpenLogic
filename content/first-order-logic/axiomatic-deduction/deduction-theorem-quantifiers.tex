% Part: first-order-logic 
% Chapter: axiomatic-deduction 
% Section: deduction-theorem-quantifiers

\documentclass[../../../include/open-logic-section]{subfiles}

\begin{document}

\olfileid{fol}{axd}{ddq}

\olsection{The Deduction Theorem with Quantifiers}

\begin{thm}[Deduction Theorem]
\ollabel{thm:deduction-thm-q} If $\Gamma \cup \{!A\} \Proves !B$,
then $\Gamma \Proves !A \lif !B$.
\end{thm}

\begin{proof}
We again proceed by induction on the length
of the !!{derivation} of $!B$ from $\Gamma \cup \{!A\}$.

The proof of the induction basis is identical to that in the proof
of~\olref[ded]{thm:deduction-thm}.

For the inductive step, suppose again that the !!{derivation} of $!B$
from $\Gamma \cup \{!A\}$ ends with a step~$!B$ which is justified by
an inference rule. If the inference rule is modus ponens, we proceed
as in the proof of \olref[ded]{thm:deduction-thm}. If the inference
rule is \QR, we know that $!B \ident !C \lif \lforall[x][!D(x)]$ and
!!a{formula} of the form $!C \lif !D(a)$ appears earlier in the
!!{derivation}, where $a$ does not occur in~$!C$, $!A$, or $\Gamma$. We
thus have that
\begin{align*}
  \Gamma \cup \{!A\} & \Proves !C \lif !D(a),\\
  \intertext{and the induction hypothesis applies, i.e., we have that}
    \Gamma & \Proves !A \lif (!C \lif !D(a)).\\
  \intertext{By}
  & \Proves (!A \lif (!C \lif !D(a))) \lif ((!A \land !C) \lif !D(a))\\
  \intertext{and modus ponens we get}
  \Gamma & \Proves (!A \land !C) \lif !D(a).\\
  \intertext{Since the eigenvariable condition still applies, we can add a step to this !!{derivation} justified by \QR, and get}
    \Gamma & \Proves (!A \land !C) \lif \lforall[x][!D(x)].\\
    \intertext{We also have}
    & \Proves ((!A \land !C) \lif \lforall[x][!D(x)]) \lif (!A \lif (!C \lif \lforall[x][!D(x)]),\\
    \intertext{so by modus ponens,}
    \Gamma & \Proves !A \lif (!C \lif \lforall[x][!D(x)]),
\end{align*}
i.e., $\Gamma \Proves !B$.

We leave the case where $!B$ is justified by the rule \QR, but is of
the form $\lexists[x][!D(x)] \lif !C$, as an exercise.
\end{proof}

\begin{prob}
  Complete the proof of \olref[fol][axd][ddq]{thm:deduction-thm-q}.
\end{prob}

\end{document}
