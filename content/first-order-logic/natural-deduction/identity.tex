% Part: first-order-logic
% Chapter: natural-deduction
% Section: identity

\documentclass[../../../include/open-logic-section]{subfiles}

\begin{document}

\olfileid{fol}{ntd}{ide}

\olsection{\usetoken{P}{derivation} with \usetoken{S}{identity}}

!!^{derivation}s with !!{identity} require additional inference rules.

\begin{defish}
\AxiomC{}
\RightLabel{\Intro{\eq}}
\UnaryInfC{$\eq[t][t]$}
\DisplayProof
\hfill
\begin{tabular}{r}
\AxiomC{$\eq[t_1][t_2]$}
\AxiomC{$!A(t_1)$}
\RightLabel{\Elim{\eq}}
\BinaryInfC{$!A(t_2)$}
\DisplayProof
\\[3ex]
\AxiomC{$\eq[t_1][t_2]$}
\AxiomC{$!A(t_2)$}
\RightLabel{\Elim{\eq}}
\BinaryInfC{$!A(t_1)$}
\DisplayProof
\end{tabular}
\end{defish}

In the above rules, $t$, $t_1$, and $t_2$ are closed terms. The
\Intro{\eq} rule allows us to !!{derive} any identity statement of the
form $\eq[t][t]$ outright, from no assumptions.

\begin{ex}
If $s$ and $t$ are closed terms, then $!A(s), \eq[s][t] \Proves !A(t)$:
\begin{prooftree}
\AxiomC{$\eq[s][t]$}
\AxiomC{$!A(s)$}
\RightLabel{$\Elim{\eq}$}
\BinaryInfC{$!A(t)$}
\end{prooftree}
This may be familiar as the ``principle of substitutability of
identicals,'' or Leibniz' Law.
\end{ex}

\begin{prob}
Prove that $=$ is both symmetric and transitive, i.e., give
!!{derivation}s of $\lforall[x][\lforall[y][(\eq[x][y] \lif
    \eq[y][x])]]$ and $\lforall[x][\lforall[y][\lforall[z]((\eq[x][y]
    \land \eq[y][z]) \lif \eq[x][z])]]$
\end{prob}

\begin{ex}
We !!{derive} the !!{sentence}
\begin{align*}
& \lforall[x][\lforall[y][((!A(x) \land !A(y)) \lif \eq[x][y])]]
\intertext{from the !!{sentence}}
& \lexists[x][\lforall[y][(!A(y) \lif \eq[y][x])]]
\end{align*}
We develop the !!{derivation} backwards:
\begin{prooftree}
\AxiomC{$\lexists[x][\lforall[y][(!A(y) \lif \eq[y][x])]]
\quad \Discharge{!A(a) \land !A(b)}{1}$}
\DeduceC{$\eq[a][b]$}
\DischargeRule{\Intro{\lif}}{1}
\UnaryInfC{$((!A(a) \land !A(b)) \lif \eq[a][b])$}
\RightLabel{\Intro{\lforall}}
\UnaryInfC{$\lforall[y][((!A(a) \land !A(y)) \lif \eq[a][y])]$}
\RightLabel{\Intro{\lforall}}
\UnaryInfC{$\lforall[x][\lforall[y][((!A(x) \land !A(y)) \lif \eq[x][y])]]$}
\end{prooftree}
We'll now have to use the main assumption: since it is an existential
!!{formula}, we use \Elim{\lexists} to !!{derive} the intermediary
conclusion $\eq[a][b]$.
\begin{prooftree}
\AxiomC{$\lexists[x][\lforall[y][(!A(y) \lif \eq[y][x])]]$}
\AxiomC{$\Discharge{\lforall[y][(!A(y) \lif \eq[y][c]])}{2}$}
\noLine
\UnaryInfC{$\Discharge{!A(a) \land !A(b)}{1}$}
\DeduceC{$\eq[a][b]$}
\DischargeRule{\Elim{\lexists}}{2}
\BinaryInfC{$\eq[a][b]$}
\DischargeRule{\Intro{\lif}}{1}
\UnaryInfC{$((!A(a) \land !A(b)) \lif \eq[a][b])$}
\RightLabel{\Intro{\lforall}}
\UnaryInfC{$\lforall[y][((!A(a) \land !A(y)) \lif \eq[a][y])]$}
\RightLabel{\Intro{\lforall}}
\UnaryInfC{$\lforall[x][\lforall[y][((!A(x) \land !A(y)) \lif \eq[x][y])]]$}
\end{prooftree}
The sub-!!{derivation} on the top right is completed by using its
assumptions to show that $\eq[a][c]$ and $\eq[b][c]$. This requies two
separate !!{derivation}s. The derivation for $\eq[a][c]$ is as
follows:
\begin{prooftree}
\AxiomC{$\Discharge{\lforall[y][(!A(y) \lif \eq[y][c]])}{2}$}
\RightLabel{\Elim{\lforall}}
\UnaryInfC{$!A(a) \lif \eq[a][c]$}
\AxiomC{$\Discharge{!A(a) \land !A(b)}{1}$}
\RightLabel{\Elim{\land}}
\UnaryInfC{$!A(a)$}
\RightLabel{\Elim{\lif}}
\BinaryInfC{$\eq[a][c]$}
\end{prooftree}
From $\eq[a][c]$ and $\eq[b][c]$ we !!{derive} $\eq[a][b]$ by
\Elim{\eq}.
\end{ex}

\begin{prob}
Give !!{derivation}s of the following !!{formula}s:
\begin{enumerate}
\item $\lforall[x][\lforall[y][((\eq[x][y] \land !A(x)) \lif !A(y))]]$
\item $\lexists[x][!A(x)] \land \lforall[y][\lforall[z][((!A(y) \land
    !A(z)) \lif \eq[y][z])]] \lif \lexists[x][(!A(x) \land
  \lforall[y][(!A(y) \lif \eq[y][x])])]$
\end{enumerate}
\end{prob}

\end{document}
