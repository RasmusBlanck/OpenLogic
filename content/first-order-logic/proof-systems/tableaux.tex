% Part: first-order-logic
% Chapter: proof-systems
% Section: tableaux

\documentclass[../../../include/open-logic-section]{subfiles}

\begin{document}

\iftag{FOL}
      {\olfileid{fol}{prf}{tab}}
      {\olfileid{pl}{prf}{tab}}

\olsection{\usetoken{P}{tableau}}

While many !!{derivation} systems operate with arrangements of
!!{sentence}s, !!{tableau}s operate with !!{signed formula}s.
!!^a{signed formula} is a pair consisting of a truth value sign
($\True$ or $\False$) and a !!{sentence}
\[
\sFmla{\True}{!A} \text{ or } \sFmla{\False}{!A}.
\]
!!^a{tableau} consists of !!{signed formula}s arranged in a
downward-branching tree. It begins with a number of \emph{assumptions}
and continues with !!{signed formula}s which result from one of the
!!{signed formula}s above it by applying one of the rules of
inference. Each rule allows us to add one or more !!{signed formula}s
to the end of a branch, or two !!{signed formula}s side by side---in
this case a branch splits into two, with the two added !!{signed
  formula}s forming the ends of the two branches.

A rule applied to a complex !!{signed formula} results in the addition
of !!{signed formula}s which are immediate sub-!!{formula}s. They come
in pairs, one rule for each of the two signs.  For instance, the
$\TRule{\True}{\land}$ rule applies to $\sFmla{\True}{!A \land !B}$,
and allows the addition of both the two !!{signed formula}s
$\sFmla{\True}{!A}$ and~$\sFmla{\True}{!B}$ to the end of any branch
containing $\sFmla{\True}{!A \land !B}$, and the rule
$\TRule{\False}{!A \land !B}$ allows a branch to be split by adding
$\sFmla{\False}{!A}$ and $\sFmla{\False}{!B}$ side-by-side.
!!^a{tableau} is closed if every one of its branches contains a
matching pair of !!{signed formula}s $\sFmla{\True}{!A}$ and
$\sFmla{\False}{!A}$.

The $\Proves$ relation based on !!{tableau}s is defined as follows:
$\Gamma \Proves !A$ iff there is some finite set~$\Gamma_0 = \{!B_1,
\dots, !B_n\} \subseteq \Gamma$ such that there is a closed !!{tableau}
for the assumptions
\[
\{\sFmla{\False}{!A}, \sFmla{\True}{!B_1}, \dots, \sFmla{\True}{!B_n}\} 
\]
For instance, here is a closed !!{tableau} that shows that $\Proves
(!A \land !B) \lif !A$:
\begin{oltableau}
  [\sFmla{\False}{(\formula{A} \land \formula{B}) \lif \formula{A}}, just = \TAss
    [\sFmla{\True}{\formula{A} \land \formula{B}}, just = {\TRule{\False}{\lif}[1]}
      [\sFmla{\False}{\formula{A}}, just = {\TRule{\False}{\lif}[1]}
        [\sFmla{\True}{\formula{A}}, just = {\TRule{\True}{\lif}[2]}
          [\sFmla{\True}{\formula{B}}, just = {\TRule{\True}{\lif}[2]}, close]
        ]
      ]
    ]
  ]
\end{oltableau}

A set $\Gamma$ is inconsistent in the !!{tableau} calculus if there is
a closed !!{tableau} for assumptions
\[
\{\sFmla{\True}{!B_1}, \dots, \sFmla{\True}{!B_n}\} 
\]
for some $!B_i \in \Gamma$.

!!^{tableau}s were invented in the 1950s independently by Evert
Beth and Jaakko Hintikka, and simplified and popularized by Raymond
Smullyan. They are very easy to use, since constructing !!a{tableau} is a
very systematic procedure. Because of the systematic nature of
!!{tableau}s, they also lend themselves to implementation by
computer. However, !!a{tableau} is often hard to read and their
connection to proofs are sometimes not easy to see. The approach is
also quite general, and many different logics have !!{tableau}
systems. !!^{tableau}s also help us to find !!{structure}s that
satisfy given (sets of) !!{sentence}s: if the set is satisfiable, it
won't have a closed !!{tableau}, i.e., any !!{tableau} will have an
open branch. The satisfying !!{structure} can be ``read off'' an open
branch, provided every rule it is possible to apply has been applied
on that branch.  There is also a very close connection to the sequent
calculus: essentially, a closed !!{tableau} is a condensed
!!{derivation} in the sequent calculus, written upside-down.

\end{document}
