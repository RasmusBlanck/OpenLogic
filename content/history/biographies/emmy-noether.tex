% Part: history
% Chapter: biographies 
% Section: emmy-noether

\documentclass[../../../include/open-logic-section]{subfiles}

\begin{document}

\olfileid{his}{bio}{noe}

\olsection{Emmy Noether}

\olphoto{noether-emmy}{Emmy Noether}

Emmy Noether (\textsc{ner}-ter) was born in Erlangen, Germany, on
March 23, 1882, to an upper-middle class scholarly family. Hailed as
the ``mother of modern algebra,'' Noether made groundbreaking
contributions to both mathematics and physics, despite significant
barriers to women's education. In Germany at the time, young girls
were meant to be educated in arts and were not allowed to attend
college preparatory schools.  However, after auditing classes at the
Universities of G\"{o}ttingen and Erlangen (where her father was
professor of mathematics), Noether was eventually able to enroll as a
student at Erlangen in 1904, when their policy was updated to allow
female students. She received her doctorate in mathematics in 1907.

Despite her qualifications, Noether experienced much resistance during
her career. From 1908--1915, she taught at Erlangen without
pay. During this time, she caught the attention of David Hilbert, one
of the world's foremost mathematicians of the time, who invited her to
G\"{o}ttingen. However, women were prohibited from obtaining
professorships, and she was only able to lecture under Hilbert's name,
again without pay. During this time she proved what is now known as
Noether's theorem, which is still used in theoretical physics
today. Noether was finally granted the right to teach in 1919.
Hilbert's response to continued resistance of his university
colleagues reportedly was: ``Gentlemen, the faculty senate is not a
bathhouse.''

In the later 1920s, she concentrated on work in abstract algebra, and
her contributions revolutionized the field.  In her proofs she often
made use of the so-called ascending chain condition, which states that
there is no infinite strictly increasing chain of certain sets. For
instance, certain algebraic structures now known as Noetherian rings
have the property that there are no infinite sequences of ideals $I_1
\subsetneq I_2 \subsetneq \dots$.  The condition can be generalized to
any partial order (in algebra, it concerns the special case of ideals
ordered by the subset relation), and we can also consider the dual
descending chain condition, where every strictly \emph{de}creasing
sequence in a partial order eventually ends.  If a partial order
satisfies the descending chain condition, it is possible to use
induction along this order in a similar way in which we can use
induction along the $<$ order on~$\Nat$.  Such orders are called
\emph{well-founded} or \emph{Noetherian}, and the corresponding proof
principle \emph{Noetherian induction}.

Noether was Jewish, and when the Nazis came to power in 1933, she was
dismissed from her position. Luckily, Noether was able to emigrate to
the United States for a temporary position at Bryn Mawr,
Pennsylvania. During her time there she also lectured at Princeton,
although she found the university to be unwelcoming to women
\citep[81]{Dick1981}. In 1935, Noether underwent an operation to
remove a uterine tumour. She died from an infection as a result of the
surgery, and was buried at Bryn Mawr.

\begin{reading} 
For a biography of Noether, see \citet{Dick1981}.  The Perimeter
Institute for Theoretical Physics has their lectures on Noether's life
and influence available online \citep{Perimeter2015}.  If you're tired
of reading, \emph{Stuff You Missed in History Class} has a podcast on
Noether's life and influence \citep{Frey2015}.  The collected works of
Noether are available in the original German \citep{Noether1983}.
\end{reading}

\end{document}
