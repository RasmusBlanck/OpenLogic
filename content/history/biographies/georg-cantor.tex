% Part: history
% Chapter: biographies 
% Section: georg-cantor

\documentclass[../../../include/open-logic-section]{subfiles}

\begin{document}

\olfileid{his}{bio}{can}

\olsection{Georg Cantor}

\olphoto{cantor-georg}{Georg Cantor}

An early biography of Georg Cantor (\textsc{gay}-org
\textsc{kahn}-tor) claimed that he was born and found on a ship that
was sailing for Saint Petersburg, Russia, and that his parents were
unknown. This, however, is not true; although he was born in Saint
Petersburg in 1845.

Cantor received his doctorate in mathematics at the University of Berlin in
1867. He is known for his work in set theory, and is credited with founding
set theory as a distinctive research discipline.
He was the first to prove that there are infinite sets of different sizes.
His theories, and especially his theory of infinities, caused much debate
among mathematicians at the time, and his work was controversial.

Cantor's religious beliefs and his mathematical work were inextricably
tied; he even claimed that the theory of transfinite numbers had been
communicated to him directly by God. In later life, Cantor suffered
from mental illness. Beginning in 1894, and more frequently towards
his later years, Cantor was hospitalized.  The heavy criticism of his
work, including a falling out with the mathematician Leopold
Kronecker, led to depression and a lack of interest in
mathematics. During depressive episodes, Cantor would turn to
philosophy and literature, and even published a theory that Francis Bacon
was the author of Shakespeare's plays.

Cantor died on January 6, 1918, in a sanatorium in Halle.

\begin{reading} 
For full biographies of Cantor, see \citet{Dauben1990} and
\citet{Grattan-Guinness1971}.  Cantor's radical views are also
described in the BBC Radio 4 program \emph{A Brief History of
  Mathematics} \citep{Sautoy2014}.  If you'd like to hear about
Cantor's theories in rap form, see \citet{Rose2012}.
\end{reading}

\end{document}
