% part: intuitionistic-logic
% chapter: introduction
% section: bhk-interpretation

\documentclass[../../../include/open-logic-section]{subfiles}

\begin{document}

\olfileid{int}{int}{bhk}

\olsection{The Brouwer-Heyting-Kolmogorov Interpretation}

\begin{editorial}
  Proofs of validity of intuitionistic propositions using the BHK
  interpretation are confusing; they have to be explained better.
\end{editorial}

There is an informal constructive interpretation of the intuitionist
connectives, usually known as the Brouwer-Heyting-Kolmogorov
interpretation. It uses the notion of a ``construction,'' which you
may think of as a constructive proof. (We don't use ``proof'' in the
BHK interpretation so as not to get confused with the notion of
!!a{derivation} in a formal proof system.) Based on this intuitive
notion, the BHK interpretation explains the meanings of the
intuitionistic connectives.

\begin{enumerate}
\item We assume that we know what constitutes a construction of an atomic
  statement.
\item A construction of $!A_1 \land !A_2$ is a pair $\tuple{M_1, M_2}$
  where $M_1$ is a construction of~$!A_1$ and $M_2$ is a construction
  of $A_2$.
\item A construction of $!A_1 \lor !A_2$ is a pair $\tuple{s, M}$
  where $s$ is~$1$ and $M$ is a construction of~$!A_1$, or $s$ is~$2$
  and $M$ is a construction of~$!A_2$.
\item A construction of $!A \lif !B$ is a function that converts a
  construction of~$!A$ into a construction of~$!B$.
\item There is no construction for~$\lfalse$ (absurdity).
\item $\lnot !A$ is defined as synonym for $!A \lif \lfalse$. That is,
  a construction of $\lnot !A$ is a function converting a construction
  of~$!A$ into a construction of~$\lfalse$.
\end{enumerate}

\begin{ex}
Take $\lnot \lfalse$ for example. A construction of it is a function
which, given any construction of $\lfalse$ as input, provides a
construction of~$\lfalse$ as output. Obviously, the identity
function~$\Id{}$ is such a construction: given a construction~$M$
of~$\lfalse$, $\Id{}(M) = M$ yields a construction of~$\lfalse$.
\end{ex}

Generally speaking, $\lnot !A$ means ``A construction of $!A$ is impossible''.

\begin{ex}
Let us prove $!A \lif \lnot \lnot !A$ for any proposition~$!A$, which
is $!A \lif ((!A \lif \lfalse) \lif \lfalse)$. The construction should
be a function~$f$ that, given a construction~$M$ of $!A$, returns a
construction~$f(M)$ of $(!A \lif \lfalse) \lif \lfalse$. Here is
how~$f$ constructs the construction of $(!A \lif \lfalse) \lif
\lfalse$: We have to define a function $g$ which, when given a
construction~$h$ of $!A \lif \lfalse$ as input, outputs a construction
of~$\lfalse$. We can define $g$ as follows: apply the input~$h$ to the
construction~$M$ of $!A$ (that we received earlier). Since the
output~$h(M)$ of $h$ is a construction of $\lfalse$, $f(M)(h) = h(M)$ is a
construction of~$\lfalse$ if $M$ is a construction of~$!A$.
\end{ex}

\begin{ex}
Let us give a construction for $\lnot(!A \land \lnot !A)$, i.e., $(!A
\land (!A \lif \lfalse)) \lif \lfalse$. This is a function~$f$ which,
given as input a construction~$M$ of $!A \land (!A \lif \lfalse)$,
yields a construction of~$\lfalse$. A construction of a conjunction
$!B_1 \land !B_2$ is a pair $\tuple{N_1, N_2}$ where $N_1$ is a
construction of~$!B_1$ and $N_2$ is a construction of~$!B_2$. We can
define functions $p_1$ and $p_2$ which recover from a construction of
$!B_1 \land !B_2$ the constructions of $!B_1$ and $!B_2$, respectively:
\begin{align*}
  p_1(\tuple{N_1, N_2}) &= N_1\\
  p_2(\tuple{N_1, N_2}) &= N_2
\end{align*}
Here is what $f$ does: First it applies $p_1$ to its input~$M$. That
yields a construction of~$!A$. Then it applies $p_2$ to~$M$, yielding
a construction of~$!A \lif \lfalse$. Such a construction, in turn, is
a function~$p_2(M)$ which, if given as input a construction of~$!A$,
yields a construction of~$\lfalse$. In other words, if we apply
$p_2(M)$ to $p_1(M)$, we get a construction of~$\lfalse$.  Thus, we
can define $f(M) = p_2(M)(p_1(M))$.
\end{ex}

\begin{ex}
Let us give a construction of $((!A \land !B) \lif !C) \lif (!A \lif
(!B \lif !C))$, i.e., a function $f$ which turns a construction~$g$ of
$(!A \land !B) \lif !C$ into a construction of $(!A \lif (!B \lif
!C))$. The construction $g$ is itself a function (from constructions
of $!A \land !B$ to constructions of~$C$). And the output $f(g)$ is a
function~$h_g$ from constructions of $!A$ to functions from constructions
of~$!B$ to constructions of~$!C$.

Ok, this is confusing.  We have to construct a certain function $h_g$,
which will be the output of $f$ for input~$g$. The input of~$h_g$ is a
construction~$M$ of~$!A$. The output of $h_g(M)$ should be a
function~$k_M$ from constructions~$N$ of~$!B$ to constructions
of~$!C$. Let $k_{g,M}(N) = g(\tuple{M, N})$. Remember that $\tuple{M,
  N}$ is a construction of~$!A \land !B$. So $k_{g,M}$ is a
construction of $!B \lif !C$: it maps constructions~$N$ of~$!B$ to
constructions of~$!C$.  Now let $h_g(M) = k_{g,M}$. That's a function
that maps constructions~$M$ of~$!A$ to constructions $k_{g,M}$ of $!B
\lif !C$. Now let $f(g) = h_g$. That's a function that maps
constructions $g$ of $(!A \land !B) \lif !C$ to constructions of $!A
\lif (!B \lif !C)$. Whew!{}
\end{ex}

The statement $!A \lor \lnot !A$ is called the Law of Excluded
Middle. We can prove it for some specific $!A$ (e.g., $\lfalse \lor
\lnot \lfalse$), but not in general. This is because the
intuitionistic disjunction requires a construction of one of the
disjuncts, but there are statements which currently can neither be
proved nor refuted (say, Goldbach's conjecture). However, you can't
refute the law of excluded middle either: that is, $\lnot \lnot (!A
\lor \lnot !A)$ holds.

\begin{ex}
To prove $\lnot \lnot (!A \lor \lnot !A)$, we need a function~$f$ that
transforms a construction of $\lnot(!A \lor \lnot !A)$, i.e., of $(!A
\lor (!A \lif \lfalse)) \lif \lfalse$, into a construction
of~$\lfalse$.  In other words, we need a function $f$ such that $f(g)$
is a construction of~$\lfalse$ if $g$ is a construction of $\lnot(!A
\lor \lnot !A)$.

Suppose $g$ is a construction of $\lnot(!A \lor \lnot !A)$, i.e., a
function that transforms a construction of $!A \lor \lnot !A$ into a
construction of~$\lfalse$.  A construction of $!A \lor \lnot !A$ is a
pair $\tuple{s, M}$ where either $s = 1$ and $M$ is a construction of
$!A$, or $s = 2$ and $M$ is a construction of $\lnot !A$. Let $h_1$ be
the function mapping a construction~$M_1$ of $!A$ to a construction of
$!A \lor \lnot !A$: it maps $M_1$ to $\tuple{1, M_2}$. And let $h_2$
be the function mapping a construction~$M_2$ of $\lnot !A$ to a
construction of $!A \lor \lnot !A$: it maps $M_2$ to $\tuple{2, M_2}$.

Let $k$ be $\comp{h_1}{g}$: it is a function which, if given a
construction of~$!A$, returns a construction of~$\lfalse$, i.e., it is
a construction of~$!A \lif \lfalse$ or $\lnot !A$. Now let $l$ be
$\comp{h_2}{g}$. It is a function which, given a construction of
$\lnot !A$, provides a construction of~$\lfalse$. Since $k$ is a
construction of $\lnot !A$, $l(k)$ is a construction of~$\lfalse$.

Together, what we've done is describe how we can turn a
construction~$g$ of $\lnot(!A \lor \lnot !A)$ into a construction
of~$\lfalse$, i.e., the function $f$ mapping a construction~$g$ of
$\lnot(!A \lor \lnot !A)$ to the construction $l(k)$ of~$\lfalse$ is a
construction of $\lnot\lnot(!A \lor \lnot !A)$.
\end{ex}

As you can see, using the BHK interpretation to show the
intuitionistic validity of !!{formula}s quickly becomes cumbersome
and confusing. Luckily, there are better !!{derivation} systems for
intuitionistic logic, and more precise semantic interpretations.
\end{document}
