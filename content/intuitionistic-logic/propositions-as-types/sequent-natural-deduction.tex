% part: intuitionistic-logic
% chapter: propositions-as-types
% section: sequent-natural-deduction

\documentclass[../../../include/open-logic-section]{subfiles}

\begin{document}

\olfileid{int}{pty}{snd}

\olsection{Sequent Natural Deduction}

Let us write $\Gamma \Sequent !A$ if there is a natural deduction
!!{derivation} with $\Gamma$ as !!{undischarged} assumptions and $!A$
as conclusion; or $\Sequent !A$ if $\Gamma$ is empty.

We write $\Gamma, !A_1, \dots, !A_n$ for $\Gamma \cup \{!A_1, \dots,
!A_n\}$, and $\Gamma, \Delta$ for $\Gamma \cup \Delta$.

Observe that when we have $\Gamma \Sequent !A \land !A$, meaning we
have !!a{derivation} with $\Gamma$ as !!{undischarged} assumptions and
$!A \land !A$ as end-!!{formula}, then by applying $\Elim{\land}$ at
the bottom, we can get !!a{derivation} with the same !!{undischarged}
assumptions and $!A$ as conclusion. In other words, if $\Gamma
\Sequent !A \land !B$, then $\Gamma \Sequent !A$.
\begin{prooftree}
  \Axiom$\Gamma \fCenter !A \land !B$
  \RightLabel{$\Elim{\land}$}
  \UnaryInf$\Gamma \fCenter !A$
  \DisplayProof\qquad\bottomAlignProof
  \Axiom$\Gamma \fCenter !A \land !B$
  \RightLabel{$\Elim{\land}$}
  \UnaryInf$\Gamma \fCenter !B$
\end{prooftree}
The label $\Elim{\land}$ hints at the relation with the rule of
the same name in natural deduction.

Likewise, suppose we have $\Gamma, !A \Sequent !B$, meaning we have
!!a{derivation} with !!{undischarged} assumptions $\Gamma, !A$ and
end-!!{formula} $!B$. If we apply the $\Intro{\lif}$ rule, we have
!!a{derivation} with $\Gamma$ as !!{undischarged} assumptions and $!A
\lif !B$ as the end-!!{formula}, i.e., $\Gamma \Sequent !A \lif
!B$. Note how this has made the !!{discharge} of assumptions more
explicit.
\begin{prooftree}
  \Axiom $\Gamma, !A \fCenter !B$
  \RightLabel{$\Intro{\lif}$}
  \UnaryInf $\Gamma \fCenter !A \lif !B$
\end{prooftree}

We can draw conclusions from other rules in the same fashion, which is
spelled out as follows:
\begin{gather*}
  \Axiom $\Gamma \fCenter !A$
  \Axiom $\Delta \fCenter !B$
  \RightLabel{$\Intro{\land}$}
  \BinaryInf $\Gamma,\Delta \fCenter !A \land !B$
  \DisplayProof\\
  \Axiom $\Gamma \fCenter !A \land !B$
  \RightLabel{$\Elim{\land}_1$}
  \UnaryInf $\Gamma \fCenter !A$
  \DisplayProof
  \qquad
  \Axiom $\Gamma \fCenter !A \land !B$
  \RightLabel{$\Elim{\land}_2$}
  \UnaryInf $\Gamma \fCenter !B$
  \DisplayProof
  \\
  \Axiom $\Gamma \fCenter !A$
  \RightLabel{$\Intro{\lor}_1$}
  \UnaryInf $\Gamma \fCenter !A \lor !B$
  \DisplayProof\qquad
  \Axiom $\Gamma \fCenter !B$
  \RightLabel{$\Intro{\lor}_2$}
  \UnaryInf $\Gamma \fCenter !A \lor !B$
  \DisplayProof\\
  \Axiom $\Gamma \fCenter !A \lor !B$
  \Axiom $\Delta, !A \fCenter !C$
  \Axiom $\Delta', !B \fCenter !C$
  \RightLabel{$\Elim{\lor}$}
  \TrinaryInf $\Gamma, \Delta, \Delta' \fCenter !C$
  \DisplayProof
  \\
  \Axiom $\Gamma, !A \fCenter !B$
  \RightLabel{$\Intro{\lif}$}
  \UnaryInf $\Gamma \fCenter !A \lif !B$
  \DisplayProof
  \qquad
  \Axiom $\Delta \fCenter !A \lif !B$
  \Axiom $\Gamma \fCenter !A$
  \RightLabel{$\Elim{\lif}$}
  \BinaryInf $\Gamma, \Delta \fCenter !B$
  \DisplayProof
  \\
  \Axiom $\Gamma \fCenter \lfalse$
  \RightLabel{$\FalseInt$}
  \UnaryInf $\Gamma \fCenter !A$
  \DisplayProof
\end{gather*}

Any assumption by itself is !!a{derivation} of $!A$ from~$!A$, i.e.,
we always have $!A \Sequent !A$.

\begin{prooftree}
  \AxiomC{$ $}
  \UnaryInfC{$!A \fCenter !A$}
\end{prooftree}

Together, these rules can be taken as a calculus about what natural
deduction !!{derivation}s exist. They can also be taken as a
notational variant of natural deduction, in which each step records
not only the !!{formula} !!{derive}d but also the !!{undischarged}
assumptions from which it was !!{derive}d.

\begin{prooftree}
  \Axiom$!A \fCenter !A$
  \UnaryInf $!A \fCenter !A \lor (!A \lif \lfalse)$
  \Axiom $!B \fCenter !B$
  \BinaryInf$!A, !B\lif \fCenter \lfalse$
  \UnaryInf$(!B \fCenter !A \lif \lfalse$
  \UnaryInf$( !B \fCenter !A \lor (!A \lif \lfalse)$
  \Axiom$(!B \fCenter !B$
  \BinaryInf$(!B \fCenter \lfalse$
  \UnaryInf$\fCenter !B \lif \lfalse$
\end{prooftree}
where $!B$ is short for $(!A \lor (!A \lif \lfalse))\lif \lfalse$.

\end{document}
