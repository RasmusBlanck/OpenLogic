% Part: lambda-calculus
% Chapter: lambda-definability
% Section: truth-values

\documentclass[../../../include/open-logic-section]{subfiles}

\begin{document}

\olfileid{lam}{ldf}{tvr}
\olsection{Truth Values and Relations}

We can encode truth values in the pure lambda calculus as follows:
\begin{align*}
  \fn{true} & \ident \lambd[x][\lambd[y][x]]\\
  \fn{false} & \ident \lambd[x][\lambd[y][y]]
\end{align*}

Truth values are represented as \emph{selectors}, i.e., functions that
accept two arguments and returning one of them. The truth value
$\fn{true}$ selects its first argument, and $\fn{false}$ its
second. For example, $\fn{true}\, M N$ always reduces to $M$, while
$\fn{false}\, M N$ always reduces to~$N$.

\begin{defn}
We call a relation $R \subseteq \Nat^n$ !!{lambda definable} if there is
a term~$R$ such that
\begin{align*}
  R\, \num{n_1} \dots \num{n_k} & \bred \fn{true}
  \intertext{whenever $R(n_1, \dots, n_k)$ and}
  R\, \num{n_1} \dots \num{n_k} & \bred \fn{false}
\end{align*}
otherwise.
\end{defn}

For instance, the relation $\fn{IsZero} = \{0\}$ which holds of $0$
and $0$ only, is !!{lambda definable} by
\[
  \fn{IsZero} \ident \lambd[n][n (\lambd[x][\fn{false}])\, \fn{true}].
\]
How does it work? Since Church numerals are defined as iterators
(functions which apply their first argument $n$~times to the second),
we set the initial value to be $\fn{true}$, and for every step of
iteration, we return $\fn{false}$ regardless of the result of the last
iteration.  This step will be applied to the initial value $n$
times, and the result will be $\fn{true}$ if and only if the step is
not applied at all, i.e., when $n = 0$.

On the basis of this representation of truth values, we can further
define some truth functions. Here are two, the representations of
negation and conjunction:
\begin{align*}
  \fn{Not} & \ident \lambd[x][x\, \fn{false}\, \fn{true}]\\
  \fn{And} & \ident \lambd[x][\lambd[y][xy \,\fn{false}]]
\end{align*}
The function ``$\fn{Not}$'' accepts one argument, and returns
$\fn{true}$ if the argument is $\fn{false}$, and $\fn{false}$ if the
argument is~$\fn{true}$.  The function ``$\fn{And}$'' accepts two
truth values as arguments, and should return $\fn{true}$ iff both
arguments are~$\fn{true}$. Truth values are represented as selectors
(described above), so when $x$ is a truth value and is applied to two
arguments, the result will be the first argument if $x$ is $\fn{true}$
and the second argument otherwise. Now $\fn{And}$ takes its two
arguments $x$ and $y$, and in return passes $y$ and $\fn{false}$ to
its first argument~$x$. Assuming $x$ is a truth value, the result
will evaluate to~$y$ if $x$ is $\fn{true}$, and to $\fn{false}$ if $x$
is $\fn{false}$, which is just what is desired.

Note that we assume here that only truth values are used as arguments
to $\fn{And}$. If it is passed other terms, the result (i.e., the
normal form, if it exists) may well not be a truth value.

\begin{prob}
  Define the functions $\fn{Or}$ and $\fn{Xor}$ representing the truth
  functions of inclusive and exclusive disjunction using the encoding
  of truth values as $\lambd$-terms.
\end{prob}

\end{document}
