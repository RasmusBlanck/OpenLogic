% Part: many-valued-logic
% Chapter: three-valued-logics
% Section: lukasiewicz

\documentclass[../../../include/open-logic-section]{subfiles}

\begin{document}

\olfileid{mvl}{inf}{luk}

\olsection{\L ukasiewicz logic}

\begin{editorial}
  This is a short ``stub'' of a section on infinite-valued \L
  ukasiewicz logic.
\end{editorial}

\begin{defn}\ollabel{def:lukasiewicz} Infinite-valued \L ukasiewicz
logic~$\LogLuk[\infty]$ is defined using the matrix:
\begin{enumerate}
  \item The standard propositional language $\Lang L_0$ with
  $\lnot$, $\land$, $\lor$, $\lif$.
  \item The set of truth values $V_\infty$.
  \item $1$ is the only designated value, i.e., $V^+ = \{1\}$.
  \item Truth functions are given by the following functions:
  \begin{align*}
    \tf{\lnot}[\LogLuk](x) & = 1 - x\\
    \tf{\land}[\LogLuk](x,y) & = \min(x,y)\\
    \tf{\lor}[\LogLuk](x,y) & = \max(x,y)\\
    \tf{\lif}[\LogLuk](x,y) & = \min(1,1-(x-y)) = \begin{cases}
      1 & \text{if } x \le y\\
      1-(x-y) & \text{otherwise.}
    \end{cases}
    \end{align*}
\end{enumerate}
$m$-valued \L ukasiewicz logic is defined the same, except $V = V_m$.
\end{defn}

\begin{prop}
  The logic $\LogLuk[3]$ defined by \olref[thr][luk]{def:lukasiewicz}
  is the same as $\LogLuk[3]$ defined by \olref{def:lukasiewicz}.
\end{prop}

\begin{proof}
  This can be seen by comparing the truth tables for the connectives
  given in \olref[thr][luk]{def:lukasiewicz} with the truth tables
  determined by the equations in \olref{def:lukasiewicz}:
  \begin{center}
    \begin{tabular}{c|c} 
      $\tf{\lnot}$ & \\ 
      \hline  
      $1$ & $0$ \\ 
      $1/2$ & $1/2$ \\
      $0$ & $1$ 
    \end{tabular}
    \quad
    \begin{tabular}{c|ccc} 
      $\tf{\land}[\LogLuk[3]]$ & $1$ & $1/2$ & $0$ \\ 
      \hline 
      $1$ & $1$ & $1/2$ & $0$ \\ 
      $1/2$ & $1/2$ & $1/2$ & $0$\\ 
      $0$ & $0$ & $0$ & $0$ 
    \end{tabular}
    \\[2ex]
    \begin{tabular}{c|ccc} 
      $\tf{\lor}[\LogLuk[3]]$ & $1$ & $1/2$ & $0$ \\ 
      \hline 
      $1$ & $1$ & $1$ & $1$ \\ 
      $1/2$ & $1$ & $1/2$ & $1/2$ \\
      $0$ & $1$ & $1/2$ & $0$ 
    \end{tabular}
    \quad
    \begin{tabular}{c|ccc} 
      $\tf{\lif}[\LogLuk[3]]$ & $1$ & $1/2$ & $0$ \\ 
      \hline 
      $1$ & $1$ & $1/2$ & $0$ \\ 
      $1/2$ & $1$ & $1$ & $1/2$  \\ 
      $0$ & $1$ & $1$ & $1$ 
    \end{tabular}
  \end{center} 
\end{proof}

\begin{prop}\ollabel{prop:luk-infty-m}
  If $\Gamma \Entails[\LogLuk[\infty]] !B$ then $\Gamma
  \Entails[\LogLuk[m]] !B$ for all~$m \ge 2$.
\end{prop}

\begin{proof}
  Exercise.
\end{proof}

\begin{prob}
  Prove \olref[mvl][inf][luk]{prop:luk-infty-m}.
\end{prob}

In fact, the converse holds as well.

Infinite-valued \L ukasiewicz logic is the most popular fuzzy logic.
In the fuzzy logic literature, the conditional is often defined as
$\lnot !A \lor !B$. The result would be an infinite-valued strong
Kleene logic.

\begin{prob}
  Show that $(p \lif q) \lor (q \lif p)$ is a
  tautology of~$\LogLuk[\infty]$.
\end{prob}

\end{document}