% Part: many-valued-logic
% Chapter: syntax-and-semantics
% Section: semantic-notions

\documentclass[../../../include/open-logic-section]{subfiles}

\begin{document}

\olfileid{mvl}{syn}{sem}

\olsection{Semantic Notions}

Suppose a many-valued logic $\Log L$ is given by a matrix.  Then we
can define the usual semantic notions for~$\Log L$.

\begin{defn} 
\begin{enumerate}
\item !!^a{formula} $!A$ is \emph{satisfiable} if for
  some~$\pAssign{v}$, $\pSat{v}{!A}$; it is
  \emph{unsatisfiable} if for no $\pAssign{v}$, $\pSat{v}{!A}$;
\item !!^a{formula} $!A$ is a \emph{tautology} if $\pSat{v}{!A}$ for
  all !!{valuation}s~$v$;
\item If $\Gamma$ is a set of !!{formula}s, $\Gamma \Entails !A$ (``$\Gamma$
  entails $!A$'') if and only if $\pSat{v}{!A}$ for every
  !!{valuation}~$\pAssign{v}$ for which $\pSat{v}{\Gamma}$.
\item If $\Gamma$ is a set of !!{formula}s, $\Gamma$ is
  \emph{satisfiable} if there is !!a{valuation}~$\pAssign{v}$ for which
  $\pSat{v}{\Gamma}$, and $\Gamma$ is
  \emph{unsatisfiable} otherwise.
\end{enumerate} 
\end{defn}

We have some of the same facts for these notions as we do for
the case of classical logic:

\begin{prop}
\ollabel{prop:semanticalfacts} 
\begin{enumerate} 
\item $!A$ is a tautology if and only if
  $\emptyset \Entails !A$; 
\item If $\Gamma$ is satisfiable then every finite subset of $\Gamma$
  is also satisfiable; 
\item\ollabel{def:Monotony}%
Monotony: if $\Gamma \subseteq \Delta$
  and $\Gamma \Entails !A$ then also $\Delta \Entails !A$;
\item\ollabel{def:Cut}%
Transitivity: if $\Gamma \Entails !A$ and
  $\Delta \cup \{ !A\} \Entails !B$ then $\Gamma \cup \Delta \Entails
  !B$;
\end{enumerate}
\end{prop}

\begin{proof}
Exercise.
\end{proof}

\begin{prob}
Prove \olref[mvl][syn][sem]{prop:semanticalfacts}
\end{prob}

In classical logic we can connect entailment and the conditional. For
instance, we have the validity of \emph{modus ponens}: If $\Gamma
\Entails !A$ and $\Gamma \Entails !A \lif !B$ then $\Gamma \Entails
!B$.  Another important relationship between $\Entails$ and $\lif$ in
classical logic is the semantic deduction theorem: $\Gamma \Entails !A
\lif !B$ if and only if $\Gamma \cup \{!A\} \Entails !B$. These
results \emph{do not} always hold in many-valued logics. Whether they
do depends on the truth function~$\tf{\lif}$.

\end{document}
