% Part: many-valued-logic
% Chapter: three-valued-logics
% Section: goedel

\documentclass[../../../include/open-logic-section]{subfiles}

\begin{document}

\olfileid{mvl}{thr}{god}

\olsection{G\"odel logics}

Kurt G\"odel introduced a sequence of $n$-valued logics that each
contain all !!{formula}s valid in intuitionistic logic, and are
contained in classical logic.  Here is the first interesting one:

\begin{defn}\ollabel{defn:goedel}
\emph{$3$-valued G\"odel logic}~$\LogGod$ is defined using the matrix:
\begin{enumerate}
  \item The standard propositional language $\Lang L_0$ with
  $\lfalse$, $\lnot$, $\land$, $\lor$, $\lif$.
  \item The set of truth values $V = \{\True, \Undef, \False\}$.
  \item $\True$ is the only designated value, i.e., $V^+ = \{\True\}$.
  \item For $\lfalse$, we have $\tf{\lfalse} = \False$. Truth
  functions for the remaining connectives are given by the following
  tables:
  \begin{center}
    \begin{tabular}{c|c} 
      $\tf{\lnot}[\LogGod]$ & \\ 
      \hline  
      $\True$ & $\False$ \\ 
      $\Undef$ & $\False$ \\
      $\False$ & $\True$ 
    \end{tabular}
    \quad
    \begin{tabular}{c|ccc} 
      $\tf{\land}[\LogGod]$ & $\True$ & $\Undef$ & $\False$ \\ 
      \hline 
      $\True$ & $\True$ & $\Undef$ & $\False$ \\ 
      $\Undef$ & $\Undef$ & $\Undef$ & $\False$\\ 
      $\False$ & $\False$ & $\False$ & $\False$ 
    \end{tabular}
    \\[2ex]
    \begin{tabular}{c|ccc} 
      $\tf{\lor}[\LogGod]$ & $\True$ & $\Undef$ & $\False$ \\ 
      \hline 
      $\True$ & $\True$ & $\True$ & $\True$ \\ 
      $\Undef$ & $\True$ & $\Undef$ & $\Undef$ \\
      $\False$ & $\True$ & $\Undef$ & $\False$ 
    \end{tabular}
    \quad
    \begin{tabular}{c|ccc} 
      $\tf{\lif}[\LogGod]$ & $\True$ & $\Undef$ & $\False$ \\ 
      \hline 
      $\True$ & $\True$ & $\Undef$ & $\False$ \\ 
      $\Undef$ & $\True$ & $\True$ & $\False$  \\ 
      $\False$ & $\True$ & $\True$ & $\True$ 
    \end{tabular}
  \end{center} 
\end{enumerate}
\end{defn}

You'll notice that the truth tables for $\land$ and~$\lor$ are the
same as in \L ukasiewicz and strong Kleene logic, but the truth tables
for $\lnot$ and~$\lif$ differ for each. In G\"odel logic,
$\tf{\lnot}(\Undef) = \False$. In contrast to \L ukasiewicz logic and
Kleene logic, $\tf{\lif}(\Undef, \False) = \False$; in contrast to
Kleene logic (but as in \L ukasiewicz logic), $\tf{\lif}(\Undef,
\Undef) = \True$.

As the connection to intuitionistic logic alluded to above suggests,
$\LogGod[3]$ is close to intuitionistic logic. All intuitionistic
truths are tautologies in~$\LogGod[3]$, and many classical tautologies
that are not valid intuitionistically also fail to be tautologies
in~$\LogGod[3]$. For instance, the following are not tautologies:
\begin{align*}
  & p \lor \lnot p && (p \lif q) \lif (\lnot p \lor q) \\
  & \lnot\lnot p \lif p && \lnot(p \land q) \lif (\lnot p \lor \lnot q) \\
  &&& ((p \lif q) \lif p) \lif p
\end{align*}
However, not every tautology of $\LogGod[3]$ is also intuitionistically
valid, e.g., $(p \lif q) \lor (q \lif p)$.

\begin{prob}
  Give a truth table to show that $(p \lif q) \lor (q \lif p)$ is a
  tautology of~$\LogGod[3]$.
\end{prob}

\begin{prob}
  Give truth tables that show that the following are not tautologies
  of~$\LogGod[3]$
  \begin{align*}
    & (p \lif q) \lif (\lnot p \lor q) \\
    & \lnot(p \land q) \lif (\lnot p \lor \lnot q) \\
    & ((p \lif q) \lif p) \lif p
  \end{align*}
\end{prob}

\begin{prob}
  Which of the following relations hold in G\"odel logic? Give a truth table for each.
  \begin{enumerate}
    \item $p, p \lif q \Entails q$
    \item $p \lor q, \lnot p \Entails q$
    \item $p \land q \Entails p$
    \item $p \Entails p \land p$
    \item $p \Entails p \lor q$
  \end{enumerate}
\end{prob}

\end{document}