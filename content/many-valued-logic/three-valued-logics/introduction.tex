% Part: many-valued-logic
% Chapter: three-valued-logics
% Section: introduction

\documentclass[../../../include/open-logic-section]{subfiles}

\begin{document}

\olfileid{mvl}{thr}{int}

\olsection{Introduction}

If we just add one more value~$\Undef$ to $\True$ and $\False$, we get a
three-valued logic. Even though there is only one more truth value,
the possibilities for defining the truth-functions for $\lnot$,
$\land$, $\lor$, and $\lif$ are quite numerous. Then a logic might use
any combination of these truth functions, and you also have a choice
of making only $\True$ designated, or both $\True$ and~$\Undef$.

We present here a selection of the most well-known three-valued
logics, their motivations, and some of their properties.

\end{document}
