% Part: many-valued-logic
% Chapter: three-valued-logics
% Section: lukasiewicz

\documentclass[../../../include/open-logic-section]{subfiles}

\begin{document}

\olfileid{mvl}{thr}{luk}

\olsection{\L ukasiewicz logic}

One of the first published, worked out proposals for a many-valued
logic is due to the Polish philosopher Jan \L ukasiewicz in 1921.  \L
ukasiewicz was motivated by Aristotle's sea battle problem: It seems
that, \emph{today}, the sentence ``There will be a sea battle
tomorrow'' is neither true nor false: its truth value is not yet
settled.  \L ukasiewicz proposed to introduce a third truth value,
to such ``future contingent'' sentences.
\begin{quote}
I can assume without contradiction that my presence in Warsaw at a
certain moment of next year, e.g., at noon on 21 December, is at the
present time determined neither positively nor negatively. Hence it is
possible, but not necessary, that I shall be present in Warsaw at the
given time. On this assumption the proposition ``I shall be in Warsaw
at noon on 21 December of next year,'' can at the present time be
neither true nor false. For if it were true now, my future presence in
Warsaw would have to be necessary, which is contradictory to the
assumption. If it were false now, on the other hand, my future
presence in Warsaw would have to be impossible, which is also
contradictory to the assumption. Therefore the proposition considered
is at the moment neither true nor false and must possess a third
value, different from ``0'' or falsity and ``1'' or truth. This value
we can designate by ``$\frac{1}{2}$.'' It represents ``the possible,''
and joins ``the true'' and ``the false'' as a third value.
\end{quote}
We will use $\Undef$ for \L ukasiewicz's third truth
value.\footnote{\L ukasiewicz here uses ``possible'' in a way that is
uncommon today, namely to mean possible but not necessary.}

The truth functions for the connectives $\lnot$, $\land$, and $\lor$
are easy to determine on this interpretation: the negation of a future
contingent sentence is also a future contingent sentence, so
$\tf{\lnot}(\Undef) = \Undef$.  If one conjunct of a
conjunction is undetermined and the other is true, the conjunction is
also undetermined---after all, depending on how the future contingent
conjunct turns out, the conjunction might turn out to be true, and it
might turn out to be false. So \[
    \tf{\land}(\True, \Undef) =
\tf{\land}(\Undef, \True) =
\Undef.
\]
If the other conjunct is false, however, it cannot
turn out true, so \[\tf{\land}(\False, \Undef) =
\tf{\land}(\False, \Undef) = \False.\]
The other values (if the arguments are settled truth values, $\True$
or $\False$, are like in classical logic.

For the conditional, the situation is a little trickier. Suppose $q$
is a future contingent statement. If $p$ is false, then $p \lif q$
will be true, regardless of how $q$ turns out, so we should set
$\tf{\lif}(\False, \Undef) = \True$. And if $p$ is true, then
$q \lif p$ will be true, regardless of what $q$ turns out to be, so
$\tf{\lif}(\Undef, \True) = \True$. If $p$ is true, then $p
\lif q$ might turn out to be true or false, so $\tf{\lif}(\True,
\Undef) = \Undef$. Similarly, if $p$ is false, then $q
\lif p$ might turn out to be true or false, so
$\tf{\lif}(\Undef, \False) = \Undef$. This leaves the
case where $p$ and $q$ are both future contingents. On the basis of
the motivation, we should really assign $\Undef$ in this case.
However, this would make $!A \lif !A$ \emph{not} a tautology. \L
ukasiewicz had not trouble giving up $!A \lor \lnot !A$ and $\lnot(!A
\land \lnot !A)$, but balked at giving up $!A \lif !A$. So he
stipulated $\tf{\lif}(\Undef, \Undef) =
\True$.

\begin{defn}\ollabel{def:lukasiewicz}
Three-valued \L ukasiewicz logic is defined using the matrix:
\begin{enumerate}
  \item The standard propositional language $\Lang L_0$ with
  $\lnot$, $\land$, $\lor$, $\lif$.
  \item The set of truth values $V = \{\True, \Undef, \False\}$.
  \item $\True$ is the only designated value, i.e., $V^+ = \{\True\}$.
  \item Truth functions are given by the following tables:
  \begin{center}
    \begin{tabular}{c|c} 
      $\tf{\lnot}$ & \\ 
      \hline  
      $\True$ & $\False$ \\ 
      $\Undef$ & $\Undef$ \\
      $\False$ & $\True$ 
    \end{tabular}
    \quad
    \begin{tabular}{c|ccc} 
      $\tf{\land}[\LogLuk[3]]$ & $\True$ & $\Undef$ & $\False$ \\ 
      \hline 
      $\True$ & $\True$ & $\Undef$ & $\False$ \\ 
      $\Undef$ & $\Undef$ & $\Undef$ & $\False$\\ 
      $\False$ & $\False$ & $\False$ & $\False$ 
    \end{tabular}
    \\[2ex]
    \begin{tabular}{c|ccc} 
      $\tf{\lor}[\LogLuk[3]]$ & $\True$ & $\Undef$ & $\False$ \\ 
      \hline 
      $\True$ & $\True$ & $\True$ & $\True$ \\ 
      $\Undef$ & $\True$ & $\Undef$ & $\Undef$ \\
      $\False$ & $\True$ & $\Undef$ & $\False$ 
    \end{tabular}
    \quad
    \begin{tabular}{c|ccc} 
      $\tf{\lif}[\LogLuk[3]]$ & $\True$ & $\Undef$ & $\False$ \\ 
      \hline 
      $\True$ & $\True$ & $\Undef$ & $\False$ \\ 
      $\Undef$ & $\True$ & $\True$ & $\Undef$  \\ 
      $\False$ & $\True$ & $\True$ & $\True$ 
    \end{tabular}
  \end{center} 
\end{enumerate}
\end{defn}

As can easily be seen, any !!{formula} $!A$ containing only $\lnot$,
$\land$, and $\lor$ will take the truth value~$\Undef$ if all
its !!{propositional variable}s are assigned~$\Undef$. So for
instance, the classical tautologies $p \lor \lnot p$ and $\lnot(p
\land \lnot p)$ are not tautologies in $\LogLuk[3]$, since $\pValue{v}(!A)
= \Undef$ whenever $\pAssign v(p) = \Undef$.

On !!{valuation}s where $\pAssign v(p) = \True$ or $\False$, $\pValue
v(!A)$ will coincide with its classical truth value.

\begin{prop}
  If $\pAssign v(p) \in \{\True, \False\}$ for all $p$ in~$!A$, then
  $\pValue v(!A)[\LogLuk[3]] = \pValue v(!A)[\LogCL]$.
\end{prop}

\begin{prob}\label{mvl:thr:luk:prob:luk-iff} Suppose we define
  $\pValue v(!A \liff !B) = \pValue v((!A \lif !B) \land (!B \lif
  !A))$ in~$\LogLuk[3]$. What truth table would $\liff$ have?
\end{prob}

Many classical tautologies \emph{are} also tautologies in \LogLuk[3],
e.g, $\lnot p \lif (p \lif q)$. Just like in classical logic, we can
use truth tables to verify this:
\begin{center}
\begin{tabular}{cc|cccccc}
  $p$ & $q$ & $\lnot$ & $p$ & $\lif$ & $\smash{(}p$ & $\lif$ & $q\smash{)}$ \\ \hline
  \True & \True & \False & \True & \True & \True & \True & \True \\
  \True & \Undef & \False & \True & \True & \True & \Undef & \Undef \\
  \True & \False & \False & \True & \True & \True & \False & \False \\
  \Undef & \True & \Undef & \Undef & \True & \Undef & \True & \True \\
  \Undef & \Undef & \Undef & \Undef & \True & \Undef & \True & \Undef \\
  \Undef & \False & \Undef & \Undef & \True & \Undef & \Undef & \False \\
  \False & \True & \True & \False & \True & \False & \True & \True \\
  \False & \Undef & \True & \False & \True & \False & \True & \Undef \\
  \False & \False & \True & \False & \True & \False & \True & \False \\  
\end{tabular}
\end{center}

\begin{prob}
  Show that the following are tautologies in \LogLuk[3]:
  \begin{enumerate}
    \item $p \lif (q \lif p)$
    \item\label{mvl:thr:luk:prob:luk-taut-2} $\lnot(p \land q) \liff (\lnot p \lor \lnot q)$
    \item\label{mvl:thr:luk:prob:luk-taut-3} $\lnot(p \lor q) \liff (\lnot p \land \lnot q)$
  \end{enumerate}
  (In \olref[mvl][thr][luk]{prob:luk-taut-2} and
  \olref[mvl][thr][luk]{prob:luk-taut-3}, take $!A \liff !B$ as an
  abbreviation for $(!A \lif !B) \land (!B \lif !A)$, or refer to your
  solution to \olref[mvl][thr][luk]{prob:luk-iff}.)
\end{prob}

\begin{prob}
  Show that the following classical tautologies are not tautologies in~\LogLuk[3]:
  \begin{enumerate}
    \item $(\lnot p \land p) \lif q)$
    \item $((p \lif q) \lif p) \lif p$
    \item $(p \lif (p \lif q)) \lif (p \lif q)$
  \end{enumerate}
\end{prob}

One might therefore perhaps think that although not all classical
tautologies are tautologies in~$\LogLuk[3]$, they should at least take
either the value~$\True$ or the value~$\Undef$ on every
!!{valuation}. This is not the case. A counterexample is given by
\[
  \lnot(p \lif \lnot p) \lor \lnot(\lnot p \lif p)
\]
which is $\False$ if $p$ is~$\Undef$.

\begin{prob}
  Which of the following relations hold in \L ukasiewicz logic? Give a truth table for each.
  \begin{enumerate}
    \item $p, p \lif q \Entails q$
    \item $\lnot\lnot p \Entails p$
    \item $p \land q \Entails p$
    \item $p \Entails p \land p$
    \item $p \Entails p \lor q$
  \end{enumerate}
\end{prob}

\L ukasiewicz hoped to build a logic of possibility on the basis of his
three-valued system, by introducing a one-place connective $\Diamond
!A$ (for ``$!A$ is possible'') and a corresponding $\Box !A$ (for ``$!A$
is necessary''):
\begin{center}
  \begin{tabular}{c|c} 
    $\tf{\Diamond}$ & \\ 
    \hline  
    $\True$ & $\True$ \\ 
    $\Undef$ & $\True$ \\
    $\False$ & $\False$ 
  \end{tabular}
  \quad
  \begin{tabular}{c|c} 
    $\tf{\Box}$ & \\ 
    \hline  
    $\True$ & $\True$ \\ 
    $\Undef$ & $\False$ \\
    $\False$ & $\False$ 
  \end{tabular}
\end{center}
In other words, $p$ is possible iff it is not already settled as false;
and $p$ is necessary iff it is already settled as true.

\begin{prob}
  Show that $\Box p \liff \lnot\Diamond \lnot p$ and $\Diamond p \liff
  \lnot \Box \lnot p$ are tautologies in~\LogLuk[3], extended with the
  truth tables for $\Box$ and~$\Diamond$.
\end{prob}

However, the shortcomings of this proposed modal logic soon became
evident: However things turn out, $p \land \lnot p$ can never turn out
to be true. So even if it is not now settled (and therefore
undetermined), it should count as impossible, i.e., $\lnot \Diamond(p
\land \lnot p)$ should be a tautology. However, if $\pAssign v(p) =
\Undef$, then $\pValue v(\lnot \Diamond(p \land \lnot p)) =
\Undef$. Although \L ukasiewicz was correct that two truth
values will not be enough to accommodate modal distinctions such as
possiblity and necessity, introducing a third truth value is also not
enough.

\end{document}