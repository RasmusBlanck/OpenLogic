% Part: methods
% Chapter: proofs
% Section: example-2

\documentclass[../../../include/open-logic-section]{subfiles}

\begin{document}

\olfileid{mth}{prf}{ex2}

\olsection{Another Example} 

\begin{prop}
If $A \subseteq C$, then $A \cup (C \setminus A) = C$.
\end{prop}  

\begin{proof}
  Suppose that $A \subseteq C$.  We want to show that $A \cup (C
  \setminus A) = C$.
  \begin{quote}
    We begin by observing that this is a conditional statement. It is
    tacitly universally quantified: the proposition holds for all sets
    $A$ and $C$. So $A$ and $C$ are variables for arbitrary sets. To
    prove such a statement, we assume the antecedent and prove the
    consequent.

    We continue by using the assumption that $A \subseteq C$. Let's
    unpack the definition of~$\subseteq$: the assumption means that
    all !!{element}s of~$A$ are also !!{element}s of~$C$. Let's write
    this down---it's an important fact that we'll use throughout the
    proof.
  \end{quote}
  By the definition of~$\subseteq$, since $A \subseteq C$, for all
  $z$, if $z \in A$, then $z \in C$.
  \begin{quote}
    We've unpacked all the definitions that are given to us in the
    assumption. Now we can move onto the conclusion. We want to show
    that $A \cup (C \setminus A) = C$, and so we set up a proof
    similarly to the last example: we show that every !!{element} of
    $A \cup (C \setminus A)$ is also !!a{element} of~$C$ and,
    conversely, every !!{element} of $C$ is !!a{element} of $A \cup (C
    \setminus A)$. We can shorten this to: $A \cup (C \setminus A)
    \subseteq C$ and $C \subseteq A \cup (C \setminus A)$. (Here we're
    doing the opposite of unpacking a definition, but it makes the
    proof a bit easier to read.)  Since this is a conjunction, we have
    to prove both parts. To show the first part, i.e., that every
    !!{element} of $A \cup (C \setminus A)$ is also !!a{element}
    of~$C$, we assume that $z \in A \cup (C \setminus A)$ for an
    arbitrary~$z$ and show that $z \in C$. By the definition of
    $\cup$, we can conclude that $z \in A$ or $z \in C \setminus A$
    from $z \in A \cup (C \setminus A)$. You should now be getting the
    hang of this.
    \end{quote}
  $A \cup (C \setminus A) = C$ iff $A \cup (C \setminus A) \subseteq
  C$ and $C \subseteq (A \cup (C \setminus A)$.  First we prove that
  $A \cup (C \setminus A) \subseteq C$.  Let $z \in A \cup (C
  \setminus A)$. So, either $z \in A$ or $z \in (C \setminus A)$.
  \begin{quote}
    We've arrived at a disjunction, and from it we want to prove that
    $z \in C$. We do this using proof by cases.
  \end{quote}
  Case 1: $z \in A$. Since for all $z$, if $z \in A$, $z \in C$, we
  have that $z \in C$.
  \begin{quote}
    Here we've used the fact recorded earlier which followed from the
    hypothesis of the proposition that $A \subseteq C$.  The first
    case is complete, and we turn to the second case, $z \in (C
    \setminus A)$.  Recall that $C \setminus A$ denotes the
    \emph{difference} of the two sets, i.e., the set of all
    !!{element}s of~$C$ which are not !!{element}s of~$A$.  But any
    !!{element} of $C$ not in~$A$ is in particular !!a{element} of~$C$.
  \end{quote}
  Case 2: $z \in (C \setminus A)$.  This means that $z \in C$ and $z
  \notin A$. So, in particular, $z \in C$.
  \begin{quote}
    Great, we've proved the first direction. Now for the second
    direction. Here we prove that $C \subseteq A \cup (C \setminus
    A)$.  So we assume that $z \in C$ and prove that $z \in A \cup (C
    \setminus A)$.
  \end{quote}
  Now let $z \in C$. We want to show that $z \in A$ or $z \in C
  \setminus A$.
  \begin{quote}
    Since all !!{element}s of $A$ are also !!{element}s of $C$, and $C
    \setminus A$ is the set of all things that are !!{element}s of $C$
    but not $A$, it follows that $z$ is either in $A$ or in $C
    \setminus A$.  This may be a bit unclear if you don't already know
    why the result is true.  It would be better to prove it
    step-by-step.  It will help to use a simple fact which we can
    state without proof: $z \in A$ or $z \notin A$. This is called the
    ``principle of excluded middle:'' for any statement~$p$, either
    $p$ is true or its negation is true. (Here, $p$ is the statement
    that $z \in A$.)  Since this is a disjunction, we can again use
    proof-by-cases.
  \end{quote}
  Either $z \in A$ or $z \notin A$. In the former case, $z \in A \cup
  (C \setminus A)$. In the latter case, $z \in C$ and $z \notin A$, so
  $z \in C \setminus A$.  But then $z \in A \cup (C \setminus A)$.
  \begin{quote}
    Our proof is complete: we have shown that $A \cup (C \setminus A) = C$.
    \qedhere
  \end{quote}
\end{proof}

\end{document}
