% Part: model-theory
% Chapter: models-of-arithmetic
% Section: non-standard-models

\documentclass[../../../include/open-logic-section]{subfiles}

\begin{document}

\olfileid{mod}{mar}{nst}
\section{Non-Standard Models}

\begin{explain}
We call a !!{structure} for $\Lang{L_A}$ standard if it is isomorphic
to~$\Struct{N}$. If a !!{structure} isn't isomorphic to~$\Struct{N}$,
it is called non-standard.
\end{explain}

\begin{defn}
A !!{structure}~$\Struct{M}$ for $\Lang{L_A}$ is \emph{non-standard}
if it is not isomorphic to~$\Struct{N}$. The !!{element}s $x \in
\Domain{M}$ which are equal to $\Value{\num{n}}{M}$ for some $n \in
\Nat$ are called \emph{standard numbers} (of $\Struct{M}$), and those
not, \emph{non-standard numbers}.
\end{defn}

\begin{explain}
By \olref[stm]{prop:standard-domain}, any standard !!{structure}
for~$\Lang{L_A}$ contains only standard !!{element}s. Consequently, a
non-standard !!{structure} must contain at least one non-standard
element. In fact, the existence of a non-standard !!{element}
guarantees that the !!{structure} is non-standard.
\end{explain}

\begin{prop}
If !!a{structure}~$\Struct{M}$ for $\Lang{L_A}$ contains a
non-standard number, $\Struct{M}$ is non-standard.
\end{prop}

\begin{proof}
Suppose not, i.e., suppose $\Struct{M}$ standard but contains a
non-standard number~$x$. Let $g\colon \Nat \to \Domain{M}$ be an
isomorphism. It is easy to see (by induction on~$n$) that
$g(\Value{\num{n}}{N}) = \Value{\num{n}}{M}$. In other words, $g$ maps
standard numbers of~$\Struct{N}$ to standard numbers
of~$\Struct{M}$. If $\Struct{M}$ contains a non-standard number, $g$
cannot be !!{surjective}, contrary to hypothesis.
\end{proof}

\begin{prob}
Recall that $\Th{Q}$ contains the axioms
\begin{align*}
& \lforall[x][\lforall[y][(\eq[x'][y'] \lif \eq[x][y])]] \tag{$!Q_1$}\\
& \lforall[x][\eq/[\Obj 0][x']] \tag{$!Q_2$}\\
& \lforall[x][(\eq[x][\Obj 0] \lor \lexists[y][\eq[x][y']])] \tag{$!Q_3$}
\end{align*}
Give !!{structure}s~$\Struct{M_1}$, $\Struct{M_2}$, $\Struct{M_3}$ such that
\begin{enumerate}
\item $\Sat{M_1}{!Q_1}$, $\Sat{M_1}{!Q_2}$, $\Sat/{M_1}{!Q_3}$;
\item $\Sat{M_2}{!Q_1}$, $\Sat/{M_2}{!Q_2}$, $\Sat{M_2}{!Q_3}$; and
\item $\Sat/{M_3}{!Q_1}$, $\Sat{M_3}{!Q_2}$, $\Sat{M_3}{!Q_3}$;
\end{enumerate}
Obviously, you just have to specify~$\Assign{\Obj{0}}{M_i}$ and
$\Assign{\prime}{M_i}$ for each.
\end{prob}

\begin{explain}
It is easy enough to specify non-standard !!{structure}s for
$\Lang{L_A}$. For instance, take the structure with !!{domain}~$\Int$
and interpret all non-logical symbols as usual. Since negative numbers
are not values of $\num{n}$ for any~$n$, this structure is
non-standard. Of course, it will not be a \emph{model} of arithmetic
in the sense that it makes the same sentences true
as~$\Struct{N}$. For instance, $\lforall[x][\eq/[x'][\Obj{0}]]$ is
false.  However, we can prove that non-standard models of arithmetic
exist easily enough, using the compactness theorem.
\end{explain}

\begin{prop}
Let $\Th{TA} = \Setabs{!A}{\Sat{N}{!A}}$ be the theory
of~$\Struct{N}$. $\Th{TA}$ has !!a{enumerable} non-standard model.
\end{prop}

\begin{proof}
Expand $\Lang{L_A}$ by a new !!{constant}~$c$ and consider the set of
!!{sentence}s
\[
\Gamma = \Th{TA} \cup \{\eq/[c][\num{0}], \eq/[c][\num{1}],
\eq/[c][\num{2}], \dots\}
\]
Any model~$\Struct{M^c}$ of~$\Gamma$ would contain !!a{element}~$x =
\Assign{c}{M}$ which is non-standard, since $x \neq
\Value{\num{n}}{M}$ for all $n \in \Nat$. Also, obviously,
$\Sat{M^c}{\Th{TA}}$, since $\Th{TA} \subseteq \Gamma$. If we turn
$\Struct{M^c}$ into a !!{structure}~$\Struct{M}$ for $\Lang{L_A}$
simply by forgetting about~$c$, its domain still contains the
non-standard~$x$, and also~$\Sat{M}{\Th{TA}}$. The latter is
guaranteed since $c$ does not occur in~$\Th{TA}$. So, it suffices to
show that $\Gamma$ has a model.

We use the compactness theorem to show that~$\Gamma$ has a model. If
every finite subset of~$\Gamma$ is satisfiable, so
is~$\Gamma$. Consider any finite subset $\Gamma_0 \subseteq
\Gamma$. $\Gamma_0$ includes some !!{sentence}s of~$\Th{TA}$ and some
of the form~$\eq/[c][\num{n}]$, but only finitely many. Suppose $k$ is
the largest number so that $\eq/[c][\num{k}] \in \Gamma_0$. Define
$\Struct{N_k}$ by expanding~$\Struct{N}$ to include the
interpretation~$\Assign{c}{N_k} = k+1$. $\Sat{N_k}{\Gamma_0}$: if $!A
\in \Th{TA}$, $\Sat{N_k}{!A}$ since $\Struct{N_k}$ is just
like~$\Struct{N}$ in all respects except~$c$, and $c$ does not occur
in~$!A$. And $\Sat{N_k}{\eq/[c][\num{n}]}$, since $n \le k$, and
$\Value{c}{N_k} = k+1$. Thus, every finite subset of~$\Gamma$ is
satisfiable.
\end{proof}

\end{document}
