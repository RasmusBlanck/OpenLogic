% Part: normal-modal-logic
% Chapter: filtrations
% Section: preliminaries

\documentclass[../../../include/open-logic-section]{subfiles}

\begin{document}

\olfileid{nml}{fil}{pre}

\olsection{Preliminaries}

Filtrations allow us to establish the decidability of our systems of
modal logic by showing that they have the \emph{finite model
  property}, i.e., that any !!{formula} that is true (false) in a
model is also true (false) in a \emph{finite} model.  Filtrations are
defined relative to sets of !!{formula}s which are closed under
subformulas.

\begin{defn}\ollabel{defn:modallyclosed}
  A set $\Gamma$ of !!{formula}s is \emph{closed under subformulas} if it
  contains every subformula of !!a{formula} in $\Gamma$. Further,
  $\Gamma$ is \emph{modally closed} if it is closed under subformulas
  and moreover $!A \in \Gamma$ implies $\Box!A,
  \Diamond!A \in \Gamma$. 
\end{defn}

For instance, given a !!{formula}~$!A$, the set of all its
sub-!!{formula}s is closed under sub-!!{formula}s. When we're defining
a filtration of a model through the set of sub-!!{formula}s of~$!A$,
it will have the property we're after: it makes $!A$ true (false) iff
the original model does.

The set of worlds of a filtration of~$\mModel{M}$ through~$\Gamma$ is
defined as the set of all equivalence classes of the following
equivalence relation.

\begin{defn}
Let $\mModel{M} =\tuple{W, R, V}$ and suppose $\Gamma$ is closed under
sub-!!{formula}s. Define a relation $\equiv$ on $W$ to hold of any two
worlds that make the same !!{formula}s from $\Gamma$ true, i.e.:
\[
u \equiv v \quad \text{if and only if }\quad \forall !A \in \Gamma : \mSat{M}{!A}[u] \Leftrightarrow \mSat{N}{!A}[v].
\]
The equivalence class~$[w]_\equiv$ of a world~$w$, or $[w]$ for short,
is the set of all worlds $\equiv$-equivalent to~$w$:
\[
[w] = \Setabs{v}{v \equiv w}.
\]
\end{defn}

\begin{prop}
  Given $\mModel{M}$ and $\Gamma$, $\equiv$ as defined above is an
  equivalence relation, i.e., it is reflexive, symmetric, and
  transitive.
\end{prop}

\begin{proof}
  The relation $\equiv$ is reflexive, since $w$ makes exactly the same
  !!{formula}s from~$\Gamma$ true as itself. It is symmetric since if
  $u$ makes the same !!{formula}s from~$\Gamma$ true as $v$, the same
  holds for $v$ and~$u$. It is also transitive, since if $u$ makes the
  same !!{formula}s from~$\Gamma$ true as~$v$, and $v$ as $w$, then
  $u$ makes the same !!{formula}s from~$\Gamma$ true as~$w$.
\end{proof}

The relation $\equiv$, like any equivalence relation, divides~$W$ into
\emph{partitions}, i.e., subsets of~$W$ which are pairwise disjoint,
and together cover all of~$W$. Every $w \in W$ is !!a{element} of one
of the partitions, namely of~$[w]$, since $w \equiv w$. So the
partitions $[w]$ cover all of~$W$. They are pairwise disjoint, for if
$u \in [w]$ and $u \in [v]$, then $u \equiv w$ and $u \equiv v$, and
by symmetry and transitivity, $w \equiv v$, and so $[w] = [v]$.

\end{document}
