% Part: normal-modal-logic
% Chapter: frame-correspondence
% Section: equivalence-S5

\documentclass[../../../include/open-logic-section]{subfiles}

\begin{document}

\olfileid{nml}{frd}{es5}

\olsection{Equivalence Relations and \Log{S5}}

The modal logic \Log{S5} is characterized as the set of !!{formula}s
valid on all universal frames, i.e., every world is accessible from
every world, including itself. In such a scenario, $\Box$ corresponds
to necessity and $\Diamond$ to possibility: $\Box !A$ is true if $!A$
is true at \emph{every} world, and $\Diamond !A$ is true if $!A$ is
true at \emph{some} world. It turns out that \Log{S5} can also be
characterized as the !!{formula}s valid on all reflexive, symmetric,
and transitive frames, i.e., on all \emph{equivalence relations}.

\begin{defn}
  A binary relation $R$ on $W$ is an \emph{equivalence relation} if
  and only if it is reflexive, symmetric and transitive.  A relation
  $R$ on $W$ is \emph{universal} if and only if $Ruv$ for all $u,v \in
  W$.
\end{defn}

Since \Ax{T}, \Ax{B}, and \Ax{4} characterize the reflexive,
symmetric, and transitive frames, the frames where the accessibility
relation is an equivalence relation are exactly those in which all
three !!{formula}s are valid. It turns out that the equivalence
relations can also be characterized by other combinations of
!!{formula}s, since the conditions with which we've defined
equivalence relations are equivalent to combinations of other familiar
conditions on~$R$.

\begin{prop}\ollabel{prop:equivalences}
  The following are equivalent:
  \begin{enumerate}
  \item $R$ is an equivalence relation;
  \item $R$ is reflexive and euclidean;
  \item $R$ is serial, symmetric, and euclidean;
  \item $R$ is serial, symmetric, and transitive.
  \end{enumerate}
\end{prop}

\begin{proof}
  Exercise.
\end{proof}

\begin{prob}
  Prove \olref[nml][frd][es5]{prop:equivalences} by showing:
  \begin{enumerate}
  \item If $R$ is symmetric and transitive, it is euclidean.
  \item If $R$ is reflexive, it is serial.
  \item If $R$ is reflexive and euclidean, it is symmetric.
  \item If $R$ is symmetric and euclidean, it is transitive.
  \item If $R$ is serial, symmetric, and transitive, it is reflexive.
  \end{enumerate}
  Explain why this suffices for the proof that the conditions are
  equivalent.
\end{prob}

\olref{prop:equivalences} is the semantic counterpart to
\olref[prf][prs]{prop:S5}, in that it gives an equivalent characterization of
the modal logic of frames over which $R$ is an equivalence relation (the logic
traditionally referred to as \Log{S5}).

What is the relationship between universal and equivalence relations?
Although every universal relation is an equivalence relation, clearly
not every equivalence relation is universal. However, the !!{formula}s
valid on all universal relations are exactly the same as those valid
on all equivalence relations.

\begin{prop}
  Let $R$ be an equivalence relation, and for each $w \in W$ define
  the \emph{equivalence class} of $w$ as the set $[w] = \{w'\in W :
  Rww'\}$. Then:
  \begin{enumerate}
  \item $w \in [w]$;
  \item $R$ is universal on each equivalence class $[w]$;
  \item The collection of equivalence classes partitions $W$ into mutually
    exclusive and jointly exhaustive subsets.
  \end{enumerate}
\end{prop}

\begin{prop}\ollabel{prop:S5=univ}
  !!^a{formula} $!A$ is valid in all frames $\mModel{F} =\tuple{W,R}$
  where $R$ is an equivalence relation, if and only if it is valid in
  all frames $\mModel{F} =\tuple{W,R}$ where $R$ is universal. Hence,
  the logic of universal frames is just \Log{S5}.
\end{prop}

\begin{proof}
  It's immediate to verify that a universal relation~$R$ on~$W$ is an
  equivalence. Hence, if $!A$ is valid in all frames where $R$ is an
  equivalence it is valid in all universal frames. For the other
  direction, we argue contrapositively: suppose $!B$ is !!a{formula}
  that fails at a world $w$ in a model $\mModel{M} = \tuple{W, R, V}$
  based on a frame $\tuple{W,R}$, where $R$ is an equivalence on
  $W$. So $\mSat/{M}{!B}[w]$. Define a model $\mModel{M}' = \tuple{W',
    R', V'}$ as follows:
  \begin{enumerate}
  \item $W' = [w]$;
  \item $R'$ is universal on $W'$;
  \item $V'(p) = V(p) \cap W'$. 
  \end{enumerate}
  (So the set $W'$ of worlds in $\mModel{M}'$ is represented by the
  shaded area in \olref{fig:partition}.)  It is easy to see
  that $R$ and $R'$ agree on $W'$. Then one can show by induction on
  !!{formula}s that for all $w' \in W'$: $\mSat{M'}{!A}[w']$ if and
  only if $\mSat{M}{!A}[w']$ for each $!A$ (this makes sense since $W'
  \subseteq W$). In particular, $\mSat/{M'}{!B}[w]$, and $!B$ fails in
  a model based on a universal frame.
\end{proof}

\begin{figure}[t]
  \centering
  \begin{tikzpicture}[node distance=2cm, auto, thick]
    \clip [rounded corners] (0,0) -- (6,0) -- (6,4) -- (0,4) --  cycle;
    \begin{scope}
      \clip (0,0) -- (2,0)  .. controls (1.5,1.5) and (4.5,1.5)
      .. (4,4) -- (0,4) -- (0,0) -- cycle;
      \filldraw[fill=gray!40] (0,4) -- (0,2) .. controls (1.5,1.5)
      and (4.5,1.5) .. (4.5,0) -- (6,0) -- (6,4) -- (0,4) --  cycle;
    \end{scope}
    \draw [name path=line1] (2,0) .. controls (1.5,1.5) and (4.5,1.5) .. (4,4);
    \draw [name path=line2] (0,2)  .. controls (1.5,1.5) and (4.5,1.5) .. (4.5,0);
    \path [name intersections={of=line1 and line2}];
    \draw [rounded corners, use as bounding box, very thick] (0,0)
    -- (6,0) -- (6,4) -- (0,4) --  cycle; 
    \node at (2,3) {$[w]$} ;
    \node at (1,1) {$[u]$} ;
    \node at (3,0.75) {$[v]$} ;
    \node at (4.75,2) {$[z]$} ;
  \end{tikzpicture}
  \caption{A partition of $W$ in equivalence classes.}
  \ollabel{fig:partition}
\end{figure}

\end{document}

