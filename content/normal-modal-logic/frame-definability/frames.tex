% Part: normal-modal-logic
% Chapter: frame-correspondence
% Section: frames

\documentclass[../../../include/open-logic-section]{subfiles}

\begin{document}

\olfileid{nml}{frd}{fra}

\olsection{Frames}

\begin{defn}
  A \emph{frame} is a pair $\mModel{F} = \tuple{W,R}$ where $W$ is a
  non-empty set of worlds and $R$ a binary relation on~$W$. A model
  $\mModel{M}$ is \emph{based on} a frame $\mModel{F} = \tuple{W,R}$
  if and only if $\mModel{M} = \tuple{W, R, V}$ for some valuation $V$.
\end{defn}

\begin{defn}
  If $\mModel{F}$ is a frame, we say that $!A$ is \emph{valid in
    $\mModel{F}$,} $\mModel{F} \Entails !A$, if $\mSat{M}{!A}$ for
  every model~$\mModel{M}$ based on~$\mModel{F}$.
  
  If $\mClass{F}$ is a class of frames, we say $!A$ is \emph{valid in
    $\mClass{F}$,} $\mClass{F} \Entails !A$, iff $\mModel{F} \Entails
  !A$ for every frame~$\mModel{F} \in \mClass{F}$.
\end{defn}

The reason frames are interesting is that correspondence between
schemas and properties of the accessibility relation~$R$ is at the
level of frames, \emph{not of models}. For instance, although \Ax{T}
is true in all reflexive models, not every model in which \Ax{T} is
true is reflexive. However, it \emph{is} true that not only is \Ax{T}
\emph{valid} on all reflexive \emph{frames}, also every frame in
which \Ax{T} is valid is reflexive.

\begin{rem}
Validity in a class of frames is a special case of the notion of
validity in a class of models: $\mClass{F} \Entails !A$ iff
$\mClass{C} \Entails !A$ where $\mClass{C}$ is the class of all models
based on a frame in~$\mClass{F}$.

Obviously, if !!a{formula} or a schema is valid, i.e., valid with
respect to the class of \emph{all} models, it is also valid with
respect to any class~$\mClass{F}$ of frames.
\end{rem}

\end{document}
