% Part: second-order-logic
% Chapter: sol-and-sets
% Section: power-of-continuum

\documentclass[../../../include/open-logic-section]{subfiles}

\begin{document}

\olfileid{sol}{set}{pow}

\olsection{The Power of the Continuum}

\begin{explain}
In second-order logic we can quantify over subsets of the !!{domain},
but not over sets of subsets of the !!{domain}. To do this directly,
we would need \emph{third-order} logic.  For instance, if we wanted to
state Cantor's Theorem that there is no !!{injective} function from
the power set of a set to the set itself, we might try to formulate it
as ``for every set~$X$, and every set~$P$, if $P$ is the power set
of~$X$, then not $\cardle{P}{X}$''. And to say that $P$ is the power set
of~$X$ would require formalizing that the !!{element}s of $P$ are all
and only the subsets of~$X$, so something like $\lforall[Y][(P(Y)
  \liff Y \subseteq X)]$. The problem lies in $P(Y)$: that is not
!!a{formula} of second-order logic, since only terms can be arguments
to one-place relation variables like~$P$.

We can, however, \emph{simulate} quantification over sets of sets, if
the domain is large enough.  The idea is to make use of the fact that
two-place relations~$R$ relates !!{element}s of the !!{domain} to
!!{element}s of the !!{domain}. Given such an~$R$, we can collect all
the !!{element}s to which some~$x$ is $R$-related: $\Setabs{y \in
  \Domain{M}}{R(x,y)}$ is the set ``coded by''~$x$.  Conversely, if
$Z \subseteq \Pow{\Domain{M}}$ is some collection of subsets
of~$\Domain{M}$, and there are at least as many !!{element}s
of~$\Domain{M}$ as there are sets in~$Z$, then there is also a
relation~$R \subseteq \Domain{M}^2$ such that every $Y \in Z$ is coded
by some~$x$ using~$R$.
\end{explain}

\begin{defn}
If $R \subseteq \Domain{M}^2$, then \emph{$x$ $R$-codes} $\Setabs{y
  \in \Domain{M}}{R(x, y)}$.
\end{defn}

If !!a{element}~$x \in \Domain{M}$ $R$-codes a set $Z \subseteq
\Domain{M}$, then a set $Y \subseteq \Domain{M}$ codes a set of sets,
namely the sets coded by the !!{element}s of~$Y$.  So a set $Y$ can
$R$-code $\Pow{X}$. It does so iff for every $Z \subseteq X$, some $x
\in Y$ $R$-codes~$Z$, and every $x \in Y$ $R$-codes a~$Z \subseteq X$.

\begin{prop}
The !!{formula}
  \[
  \fn{Codes}(x, R, Z) \ident \lforall[y][(Z(y) \liff
    R(x, y))]
  \]
expresses that $s(x)$ $s(R)$-codes $s(Z)$.  The
!!{formula} 
\begin{multline*}
  \fn{Pow}(Y, R, X) \ident \\
  \lforall[Z][(Z \subseteq X \lif \lexists[x][(Y(x) \land
    \fn{Codes}(x, R, Z))])] \land {} \\ \lforall[x][(Y(x) \lif
  \lforall[Z][(\fn{Codes}(x, R, Z) \lif Z \subseteq X)]]
\end{multline*}
expresses that $s(Y)$ $s(R)$-codes the power set of~$s(X)$, i.e., the
elements of $s(Y)$ $s(R)$-code exactly the subsets of $s(X)$.
\end{prop}

\begin{explain}
With this trick, we can express statements about the power set by
quantifying over the codes of subsets rather than the subsets
themselves.  For instance, Cantor's Theorem can now be expressed
by saying that there is no !!{injective} function from the domain
of any relation that codes the power set of~$X$ to~$X$ itself.
\end{explain}

\begin{prop}
The sentence
\begin{multline*}
  \lforall[X][\lforall[Y][\lforall[R][(\fn{Pow}(Y, R, X) \lif \\
      \lnot \lexists[u][(\lforall[x][\lforall[y][(\eq[u(x)][u(y)] \lif
            \eq[x][y])]] \land {}\\
        \lforall[x][(Y(x) \lif X(u(x)))])])]]]
\end{multline*}
is valid.
\end{prop}

\begin{explain}
The power set of !!a{denumerable} set is !!{nonenumerable}, and so its
cardinality is larger than that of any !!{denumerable} set (which is
$\aleph_0$).  The size of $\Pow{\Nat}$ is called the ``power of the
continuum,'' since it is the same size as the points on the real
number line,~$\Real$. If the !!{domain} is large enough to code the
power set of !!a{denumerable} set, we can express that a set is the
size of the continuum by saying that it is equinumerous with any
set~$Y$ that codes the power set of set~$X$ of size~$\aleph_0$. (If the
domain is not large enough, i.e., it contains no subset equinumerous
with~$\Real$, then there can also be no relation that codes~$\Pow{X}$.)
\end{explain}

\begin{prop}
If $\cardle{\Real}{\Domain{M}}$, then the !!{formula}
\begin{multline*}
\fn{Cont}(Y) \ident 
\lexists[X][\lexists[R][((\fn{Aleph}_0(X) \land
      \fn{Pow}(Y, R, X)) \land \\ 
      \lforall[x][\lforall[y][((Y(x) \land {} 
      Y(y) \land \lforall[z][R(x,z) \liff R(y,z)]) \lif \eq[x][y])]])]]
\end{multline*}
expresses that $\cardeq{s(Y)}{\Real}$.
\end{prop}

\begin{proof}
  $\fn{Pow}(Y, R, X)$ expresses that $s(Y)$ $s(R)$-codes the power set
  of~$s(X)$, which $\fn{Aleph}_0(X)$ says is countable. So $s(Y)$ is
  at least as large as the power of the continuum, although it may be
  larger (if multiple elements of $s(Y)$ code the same subset of~$X$).
  This is ruled out be the last conjunct, which requires the
  association between elements of $s(Y)$ and subsets of $s(Z)$ via
  $s(R)$ to be !!{injective}.
\end{proof}

\begin{prop}
$\cardeq{\Domain{M}}{\Real}$ iff
\begin{multline*}
  \Sat{M}{\lexists[X][\lexists[Y][\lexists[R][(\fn{Aleph}_0(X) \land \fn{Pow}(Y, R, X) \land \\
          \lexists[u][(\lforall[x][\lforall[y][(\eq[u(x)][u(y)] \lif \eq[x][y])]] \land {} \\\lforall[y][(Y(y) \lif \lexists[x][\eq[y][u(x)]])])])]]]}.
        \end{multline*}
  \end{prop}

\begin{explain}
The Continuum Hypothesis is the statement that the size of the
continuum is the first !!{nonenumerable} cardinality, i.e, that
$\Pow{\Nat}$ has size~$\aleph_1$. 
\end{explain}

\begin{prop}
The Continuum Hypothesis is true iff \[\fn{CH} \ident
\lforall[X][(\fn{Aleph}_1(X) \liff \fn{Cont}(X))]\] is valid.
\end{prop}

Note that it isn't true that $\lnot \fn{CH}$ is valid iff the
Continuum Hypothesis is false. In !!a{enumerable} domain, there are no
subsets of size~$\aleph_1$ and also no subsets of the size of the
continuum, so $\fn{CH}$ is always true in !!a{enumerable}
domain. However, we can give a different sentence that is valid iff
the Continuum Hypothesis is false:

\begin{prop}
The Continuum Hypothesis is false iff \[\fn{NCH} \ident
\lforall[X][(\fn{Cont}(X) \lif \lexists[Y][(Y \subseteq X \land \lnot
    \fn{Count}(Y) \land \lnot \cardeq{X}{Y})])]\] is valid.
\end{prop}

\end{document}
