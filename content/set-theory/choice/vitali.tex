\documentclass[../../../include/open-logic-section]{subfiles}

\begin{document}

\olfileid{sth}{choice}{vitali}
\olsection{Vitali's Paradox}

To get a real sense of whether the Banach-Tarski construction is
acceptable or not, we should examine its \emph{proof}. Unfortunately,
that would require much more algebra than we can present here.
However, we can offer some quick remarks which might shed some insight
on the proof of Banach-Tarski,\footnote{For a much fuller treatment,
see \cite{Weston2003} or \cite{Wagon2016}.} by focussing on the
following result:

\begin{thm}[Vitali's Paradox (in $\ZFC$)]\ollabel{vitaliparadox}
Any circle can be decomposed into countably many pieces, which can be
reassembled (by rotation and transportation) to form two copies of
that circle.
\end{thm}

This is much easier to prove than the Banach--Tarski Paradox. We have
called it ``Vitali's Paradox'', since it follows from Vitali's
\citeyear{Vitali1905} construction of an unmeasurable set.  But the
set-theoretic aspects of the proof of Vitali's Paradox and the
Banach-Tarski Paradox are very similar. The essential difference
between the results is just that Banach-Tarski considers a
\emph{finite} decomposition, whereas Vitali's Paradox onsiders a
\emph{countably infinite} decomposition.  As \citet{Weston2003}
puts it, Vitali's Paradox ``is certainly not nearly as striking as the
Banach--Tarski paradox, but it does illustrate that geometric
paradoxes can  happen even in `simple' situations.'' 

Vitali's Paradox concerns a two-dimensional figure, a circle. So we
will work on the plane, $\Real^2$. Let $\rotationsgroup$ be the set of
(clockwise) rotations of points around the origin by \emph{rational}
radian values between $[0,2\pi)$. Here are some algebraic facts about
$\rotationsgroup$ (if you don't understand the statement of the
result, the proof will make its meaning clear):

\begin{lem}\ollabel{rotationsgroupabelian}
$\rotationsgroup$ forms an abelian {group} under composition of functions.
\end{lem}

\begin{proof}
Writing $0_{\rotationsgroup}$ for the rotation by $0$ radians, this is
an identity element for $\rotationsgroup$, since
$\comp{0_{\rotationsgroup}}{\rho} = \comp{\rho}{0_{\rotationsgroup}} =
\rho$ for any $\rho \in \rotationsgroup$.

Every element has an inverse. Where $\rho \in \rotationsgroup$ rotates
by $r$ radians, $\rho^{-1} \in \rotationsgroup$ rotates by $2\pi - r$
radians, so that $\rho \circ \rho^{-1} = 0_\rotationsgroup$.

Composition is associative: $\comp{\rho}{(\comp{\sigma}{\tau})} =
\comp{(\comp{\rho}{\sigma})}{\tau}$ for any $\rho, \sigma, \tau \in
\rotationsgroup$

Composition is commutative: $\comp{\rho}{\sigma} =
\comp{\sigma}{\rho}$ for any $\rho, \sigma \in \rotationsgroup$.
\end{proof}

In fact, we can split our group $\rotationsgroup$
in half, and then use either half to recover the whole group:

\begin{lem}\ollabel{disjointgroup}
There is a partition of $\rotationsgroup$ into two disjoint sets,
$\rotationsgroup_{1}$ and $\rotationsgroup_{2}$, both of which are a
basis for $\rotationsgroup$. 
\end{lem}

\begin{proof}
Let $\rotationsgroup_{1}$ consist of the rotations by rational radian
values in $[0, \pi)$; let $\rotationsgroup_{2} = \rotationsgroup
\setminus \rotationsgroup_1$. By elementary algebra,
$\Setabs{\comp{\rho}{\rho}}{\rho \in \rotationsgroup_1} =
\rotationsgroup$. A similar result can be obtained for
$\rotationsgroup_2$.
\end{proof}

We will use this fact about groups to establish \olref{vitaliparadox}.
Let $\onesphere$ be the unit circle, i.e., the set of points
exactly~$1$ unit away from the origin of the plane, i.e.,
$\Setabs{\tuple{r,s} \in \Real^2}{\sqrt{r^2+s^2}=1}$. We will split
$\onesphere$ into parts by considering the following relation on
$\onesphere$:
\[
	r \sim s \emph{ iff }(\exists \rho \in \rotationsgroup)\rho(r) = s.
\]
That is, the points of $\onesphere$ are linked by this relation iff you can get from one to the other by a rational-valued rotation about the origin. Unsurprisingly:

\begin{lem}
$\sim$ is an equivalence relation.
\end{lem}

\begin{proof}
Trivial, using \olref{rotationsgroupabelian}. 
\end{proof}

We now invoke Choice to obtain a set, $C$, containing exactly one
member from each equivalence class of $\onesphere$ under $\sim$. That
is, we consider a choice function $f$ on the set of equivalence
classes,\footnote{Note: since $\rotationsgroup$ is !!{enumerable},
each !!{element} of $E$ is !!{enumerable}. Since $\onesphere$ is
!!{nonenumerable}, it follows from \olref{vitalicover} and
\olref[card-arithmetic][simp]{kappaunionkappasize} that $E$ is
!!{nonenumerable}. So this is a use of \emph{un}countable Choice.}
\[
	E = \Setabs{\equivrep{r}{\sim}}{r \in \onesphere},
\]
and let $C = \ran{f}$. For each rotation $\rho \in \rotationsgroup$,
the set $\funimage{\rho}{C}$ consists of the points obtained by
applying the rotation $\rho$ to each point in $C$. These next two
results show that these sets cover the circle completely and without
overlap:

\begin{lem}\ollabel{vitalicover}
$\onesphere = \bigcup_{\rho \in \rotationsgroup} \funimage{\rho}{C}$.
\end{lem}

\begin{proof}
Fix $s \in \onesphere$; there is some $r \in C$ such that $r \in
\equivrep{s}{\sim}$, i.e., $r \sim s$, i.e., $\rho(r) = s$ for some
$\rho \in \rotationsgroup$. 
\end{proof}

\begin{lem}\ollabel{vitalinooverlap}
If $\rho_1 \neq \rho_2$ then $\funimage{\rho_1}{C} \cap \funimage{\rho_2}{C} = \emptyset$. 
\end{lem}

\begin{proof}
Suppose $s \in \funimage{\rho_{1}}{C} \cap \funimage{\rho_{2}}{C}$. So
$s = \rho_{1}(r_{1}) = \rho_{2}(r_{2})$ for some $r_{1}, r_{2} \in C$.
Hence $\rho^{-1}_2(\rho_1(r_1)) = r_2$, and
$\comp{\rho_1}{\rho^{-1}_2} \in \rotationsgroup$, so $r_{1} \sim
r_{2}$. So $r_1 = r_2$, as $C$ selects exactly one member from each
equivalence class under $\sim$. So $s = \rho_1(r_1) = \rho_2(r_1)$,
and hence $\rho_1 = \rho_2$.
\end{proof}

We now apply our earlier algebraic facts to our circle:

\begin{lem}\ollabel{pseudobanachtarski}
There is a partition of $\onesphere$ into two disjoint sets, $D_{1}$
and $D_{2}$, such that $D_{1}$ can be partitioned into countably many
sets which can be rotated to form a copy of $\onesphere$ (and
similarly for $D_{2}$).
\end{lem}

\begin{proof}
Using $\rotationsgroup_{1}$ and $\rotationsgroup_{2}$ from \olref{disjointgroup}, let:
\begin{align*}
	D_{1} &= \bigcup_{\rho \in \rotationsgroup_1} \funimage{\rho}{C} & 
	D_{2} &= \bigcup_{\rho \in \rotationsgroup_1} \funimage{\rho}{C}
\end{align*}
This is a partition of $\onesphere$, by \olref{vitalicover}, and $D_1$
and $D_2$ are disjoint by \olref{vitalinooverlap}. By construction,
$D_1$ can be partitioned into countably many sets,
$\funimage{\rho}{C}$ for each $\rho \in R_1$. And these can be rotated
to form a copy of $\onesphere$, since $\onesphere = \bigcup_{\rho \in
\rotationsgroup}\funimage{\rho}{C} = \bigcup_{\rho \in
\rotationsgroup_1}\funimage{(\comp{\rho}{\rho})}{C}$ by
\olref{disjointgroup} and \olref{vitalicover}. The same reasoning
applies to $D_2$. \end{proof}\noindent This immediately entails
Vitali's Paradox. For we can generate \emph{two} copies of
$\onesphere$ from $\onesphere$, just by splitting it up into countably
many pieces (the various $\funimage{\rho}{C}$'s) and then rigidly
moving them (simply rotate each piece of $D_1$, and first transport
and then rotate each piece of $D_2$).

Let's recap the proof-strategy. We started with some algebraic facts
about the group of rotations on the plane. We used this group to
partition $\onesphere$ into equivalence classes. We then arrived at a
``paradox'', by using Choice to select elements from each class.

We use exactly the same strategy to prove Banach--Tarski. The main
difference is that the algebraic facts used to prove Banach--Tarski
are significantly more complicated than those used to prove
Vitali's Paradox. But those algebraic facts have nothing to do with
Choice. We will summarise them quickly. 

To prove Banach--Tarski, we start by establishing an analogue of
\olref{disjointgroup}: any \emph{free group} can be split into four
pieces, which intuitively we can ``move around'' to recover two copies
of the whole group.\footnote{The fact that we can use \emph{four}
pieces is due to \cite{Robinson1947}. For a recent proof, see
\citet[Theorem 5.2]{Wagon2016}. We follow \citet[p.~3]{Weston2003}
in describing this as ``moving'' the pieces of the group.} We then
show that we can use two particular rotations around the origin of
$\Real^3$ to generate a free group of rotations, $F$.\footnote{See
\citet[Theorem 2.1]{Wagon2016}.} (No Choice yet.) We now regard points
on the surface of the sphere as ``similar'' iff one can be obtained
from the other by a rotation in~$F$. We then \emph{use Choice} to
select exactly one point from each equivalence class of ``similar''
points. Applying our division of $F$ to the surface of the sphere, as
in \olref{pseudobanachtarski}, we split that surface into four pieces,
which we can ``move around'' to obtain two copies of the surface of
the sphere. And this establishes \citep{Hausdorff1914}:

\begin{thm}[Hausdorff's Paradox (in $\ZFC$)] 
The surface of any sphere can be decomposed into finitely many pieces,
which can be reassembled (by rotation and transportation) to form two
disjoint copies of that sphere.
\end{thm}

A couple of further algebraic tricks are needed to obtain the full
Banach-Tarski Theorem (which concerns not just the sphere's surface,
but its interior too). Frankly, however, this is just icing on the
algebraic cake. Hence Weston writes:
\begin{quote}	
  [\ldots] the result on free groups is the \emph{key step} in the
  proof of the Banach-Tarski paradox. From this point of view, the
  Banach-Tarski paradox is not a statement about $\Real^3$ so much as
  it is a statement about the complexity of the group [of translations
  and rotations in $\Real^3$]. \cite[p.~16]{Weston2003}
\end{quote}
That is: whether we can offer a \emph{finite} decomposition (as in
Banach--Tarski) or a \emph{countably infinite} decomposition (as in
Vitali's Paradox) comes down to certain group-theoretic facts about
working in two-dimension or three-dimensions.

Admittedly, this last observation slightly spoils the joke at the end
of \olref[banach]{sec}. Since it is  two dimensional,
``Banach-Tarski'' must be divided into a countable \emph{infinity} of
pieces, if one wants to rearrange those pieces to form ``Banach-Tarski
Banach-Tarski''. To repair the joke, one must write in three
dimensions. We leave this as an exercise for the reader.

One final comment. In \olref[banach]{sec}, we mentioned that the
``pieces'' of the sphere one obtains cannot be \emph{measurable}, but
must be unpicturable ``infinite scatterings''. The same is true of our
use of Choice in obtaining \olref{pseudobanachtarski}. And this is all
worth explaining.

Again, we must sketch some background (but this is \emph{just} a
sketch; you may want to consult a textbook entry on \emph{measure}).
To define a measure for a set $X$ is to assign a value $\mu(E) \in
\Real$ for each $E$ in some ``$\sigma$-algebra'' on $X$. Details here
are not essential, except that the function $\mu$ must obey the
principle of countable additivity: the measure of a countable union of
disjoint sets is the sum of their individual measures, i.e.,
$\mu(\bigcup_{n < \omega} X_n) = \sum_{n < \omega}\mu(X_n)$ whenever
the $X_n$s are disjoint. To say that a set is ``unmeasurable'' is to
say that no measure can be suitably assigned. Now, using our
$\rotationsgroup$ from before:

\begin{cor}[Vitali]
Let $\mu$ be a measure such that $\mu(\onesphere) = 1$, and such that
$\mu(X) = \mu(Y)$ if $X$ and $Y$ are congruent. Then
$\funimage{\rho}{C}$ is unmeasurable for all $\rho \in
\rotationsgroup$. 
\end{cor}

\begin{proof}
For reductio, suppose otherwise. So let $\mu(\funimage{\sigma}{C}) =
r$ for some $\sigma \in \rotationsgroup$ and some $r \in \Real$. For
any $\rho \in C$, $\funimage{\rho}{C}$ and $\funimage{\sigma}{C}$ are
congruent, and hence  $\mu(\funimage{\rho}{C}) = r$ for any $\rho \in
C$. By \olref{vitalicover} and \olref{vitalinooverlap}, $\onesphere =
\bigcup_{\rho \in \rotationsgroup}\funimage{\rho}{C}$ is a countable
union of pairwise disjoint sets. So countable additivity dictates that
$\mu(\onesphere) = 1$ is the sum of the measures of each
$\funimage{\rho}{C}$, i.e.,
\[
	1 = \mu(\onesphere) = \sum_{\rho \in \rotationsgroup}\mu(\funimage{\rho}{C}) = \sum_{\rho \in \rotationsgroup}r
\]
But if $r = 0$ then $\sum_{\rho \in \rotationsgroup}r = 0$, and if $r
> 0$ then $\sum_{\rho \in \rotationsgroup}r = \infty$. 
\end{proof}

\end{document}