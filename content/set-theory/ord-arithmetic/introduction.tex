\documentclass[../../../include/open-logic-section]{subfiles}

\begin{document}
\olfileid{sth}{ord-arithmetic}{intro}

\olsection{Introduction}

In \olref[ordinals][]{chap}, we developed a theory of ordinal numbers.
We saw in \olref[spine][]{chap} that we can think of the ordinals as a
spine around which the remainder of the hierarchy is constructed. But
that is not the only role for the ordinals. There is also the task of
performing ordinal arithmetic. 

We already gestured at this, back in
\olref[ordinals][idea]{sec}, when we spoke of $\omega$,
$\omega+1$ and $\omega+\omega$. At the time, we spoke informally; the
time has come to spell it out properly. However, we should mention
that there is not much philosophy in this chapter; just technical
developments, coupled with a (mildly) interesting observation that we
can do the same thing in two different ways.

\end{document}