% Part: set-theory
% Chapter: ordinals
% Section: introduction

\documentclass[../../../include/open-logic-section]{subfiles}

\begin{document}

\olfileid{sth}{ordinals}{intro}

\olsection{Introduction}

In \olref[z][]{chap}, we postulated that there is an infinite-th stage
of the hierarchy, in the form of \stagesinf{} (see also our axiom of
Infinity). However, given \stagessucc{}, we can't stop at the
infinite-th stage; we have to keep going. So: at the next stage after
the first infinite stage, we form all possible collections of sets
that were available at the first infinite stage; and repeat; and
repeat; and repeat; \dots

Implicitly what has happened here is that we have started to invoke an
``intuitive'' notion of number, according to which there can be
numbers \emph{after} all the natural numbers. In particular, the
notion involved is that of a \emph{transfinite ordinal}. The aim of
this chapter is to make this idea more rigorous. We will explore the
general notion of an ordinal, and then explicitly define certain sets
to be our ordinals. 

\end{document}