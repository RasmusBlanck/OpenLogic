\documentclass[../../../include/open-logic-section]{subfiles}

\begin{document}

\olfileid{sth}{replacement}{limofsize}
\olsection{Limitation-of-size}

Perhaps the most common to offer an ``intrinsic'' justification of
Replacement comes via the following notion:
\begin{enumerate}
	\item[] \limofsize. Any things form a set, provided that there are
	not too many of them.
\end{enumerate}
This principle will immediately vindicate Replacement. After all, any
set formed by Replacement cannot be any larger than any set from which
it was formed. Stated precisely: suppose you form a set
$\funimage{\tau}{A} = \Setabs{\tau(x)}{x \in A}$ using Replacement;
then $\cardle{\funimage{\tau}{A}}{A}$; so if the !!{element}s of $A$
were not too numerous to form a set, their images are not too numerous
to form $\funimage{\tau}{A}$. 

The obvious difficulty with invoking \limofsize{} to justify
Replacement is that we have \emph{not} yet laid down any principle
like \limofsize. Moreover, when we told our story about the
cumulative-iterative conception of set in
\crefrange{sth:story::chap}{sth:z::chap}, nothing ever \emph{hinted}
in the direction of \limofsize. This, indeed, is precisely why Boolos
at one point wrote: ``Perhaps one may conclude that there are at least
two thoughts `behind' set theory'' \citeyearpar[p.~19]{Boolos1989}. On
the one hand, the ideas surrounding the cumulative-iterative
conception of set are meant to vindicate~$\Z$. On the other hand,
\limofsize{} is meant to vindicate Replacement. 

But the issue it is not just that we have thus far been \emph{silent}
about \limofsize. Rather, the issue is that \limofsize{} (as just
formulated) seems to sit quite badly with the cumulative-iterative
notion of set. After all, it mentions nothing about the idea of sets
as formed in \emph{stages}.

This is really not much of a surprise, given the history of these
``two thoughts'' (i.e., the cumulative-iterative conception of set,
and \limofsize). These ``two thoughts'' ultimately amount to two
rather  different projects for blocking the set-theoretic paradoxes.
The cumulative-iterative notion of set blocks Russell's paradox by
saying, roughly: \emph{we should never have expected a Russell set to
exist, because it would not be ``formed'' at any stage}. By contrast,
\limofsize{} is meant to rule out the Russell set, by saying, roughly:
\emph{we should never have expected a Russell set to exist, because it
would have been too big}. 

Put like this, then, let's be blunt: considered as a reply to the
paradoxes, \limofsize{} stands in need of \emph{much} more
justification. Consider, for example, this version of Russell's
Paradox: \emph{no pug sniffs exactly the pugs which don't sniff
themselves}. If one asks ``why is there no such pug?''\ it is not a
good answer to be told that such a pug would have to sniff too many
pugs. So why would it be a good intuitive explanation, for the
non-existence of a Russell set, that it would have to be ``too big''
to exist? 

So it's forgivable if you are a bit mystified concerning the ``intuitive''
motivation for \limofsize. 

\end{document}