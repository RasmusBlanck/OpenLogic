\documentclass[../../../include/open-logic-section]{subfiles}

\begin{document}

\olfileid{sth}{spine}{valpha}

\olsection{Defining the Stages as the $V_\alpha$s}

In \olref[sth][ordinals][]{chap}, we defined well-orderings and the
(von Neumann) ordinals. In this chapter, we will use these to
characterise the hierarchy of sets \emph{itself}. To do this, recall
that in \olref[sth][ordinals][opps]{sec}, we defined the idea of
successor and limit ordinals. We use these ideas in following
definition:

\begin{defn}\ollabel{defValphas}
	\begin{align*}
	V_\emptyset &:= \emptyset\\
	V_{\ordsucc{\alpha}} &:= \Pow{V_\alpha} & & 
	\text{for any ordinal }\alpha\\
	V_{\alpha} &:= \bigcup_{\gamma < \alpha} V_\gamma & & 
	\text{when }\alpha\text{ is a limit ordinal}
\end{align*}
\end{defn}

This will be a definition by \emph{transfinite recursion} on the
ordinals. In this regard, we should compare this with recursive
definitions of functions on the natural numbers.\footnote{Cf.\ the
definitions of addition, multiplication, and exponentiation in
\olref[sfr][infinite][dedekind]{sec}.} As when dealing with natural
numbers, one defines a base case and successor cases; but when dealing
with ordinals, we also need to describe the behaviour of \emph{limit}
cases. 

This definition of the $V_\alpha$s will be an important milestone. We
have informally motivated our hierarchy of sets as forming sets by
\emph{stages}. The $V_\alpha$s are, in effect, just those stages.
Importantly, though, this is an \emph{internal} characterisation of
the stages. Rather than suggesting a possible \emph{model} of the
theory, we will have defined the stages \emph{within} our set theory.

\end{document}
