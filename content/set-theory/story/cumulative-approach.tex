\documentclass[../../../include/open-logic-section]{subfiles}

\begin{document}

\olfileid{sth}{story}{approach}
\olsection{The Cumulative-Iterative Approach}

Here is a slightly fuller statement of how we will stratify sets into
stages:
\begin{quote}
	Sets are formed in \emph{stages}. For each stage $S$, there are
	certain stages which are \emph{before} $S$. At stage $S$, each
	collection consisting of sets formed at stages before $S$ is
	formed into a set. There are no sets other than the sets which are
	formed at stages. \citep[p.~323]{Shoenfield:AST}
\end{quote} 
This is a sketch of the \emph{cumulative-iterative conception of
set}. It will underpin the formal set theory that we present in
\olref[sth][][]{part}. 

Let's explore this in a little more detail. As Shoenfield describes
the process, at every stage, we form new sets from the {sets} which
were available to us from earlier stages. So, on Shoenfield's picture,
at the initial stage, stage $0$, there are no \emph{earlier} stages,
and so \emph{a fortiori} there are no sets available to us from
earlier stages.\footnote{Why should we assume that there \emph{is} a
first stage? See the footnote to \stagesord{} in
\olref[z][story]{sec}.} So we form only one set: the set
with no !!{element}s $\emptyset$. At stage $1$, exactly one set is
available to us from earlier stages, so only one new set is
$\{\emptyset\}$. At stage $2$, two sets are available to us from
earlier stages, and we form two new sets $\{\{\emptyset\}\}$ and
$\{\emptyset, \{\emptyset\}\}$. At stage $3$, four sets are available
to us from earlier stages, so we form twelve new sets\ldots. As such,
the cumulative-iterative  picture of the sets will look a bit like
this (with numbers indicating stages):
\begin{center}
	\begin{tikzpicture}[scale=0.6]
	\tikzset{cut_here/.style={densely dotted}}
	\draw (-4,6) -- (-1, 0)-- (2,6); 
	\draw[cut_here] (-5,8)--(-4,6);
	\draw[cut_here] (2,6)--(3,8);
	\draw(-1.5, 1)--(-0.5, 1);
	\draw(-2, 2)--(0, 2);
	\draw(-2.5, 3)--(0.5,3);
	\draw(-3, 4)--(1, 4);
	\draw(-3.5, 5)--(1.5, 5);
	\draw(-4, 6)--(2, 6);
	\node[label] at (0, 0) {\small 0};
	\node[label] at (0.5, 1) {\small 1};
	\node[label] at (1, 2) {\small 2};
	\node[label] at (1.5, 3) {\small 3};
	\node[label] at (2, 4) {\small 4};
	\node[label] at (2.5, 5) {\small 5};
	\node[label] at (3, 6) {\small 6};
	\end{tikzpicture}
\end{center}
So: why should we embrace this story? 

One reason is that it is a nice, tractable story. Given the demise of
the most obvious story, i.e., Na\"ive Comprehension, we are in want of
something nice. 

But the story is not \emph{just} nice. We have a good reason to
believe that any set theory based on this story will be
\emph{consistent}. Here is why. 

Given the cumulative-iterative conception of set, we form sets at
stages; and their !!{element}s must be objects which were available
\emph{already}. So, for any stage~$S$, we can form the set 
\[
	R_S = \Setabs{x}{x \notin x \text{ and $x$ was available before $S$}}
\]
The reasoning involved in proving Russell's Paradox will now establish
that $R_S$ itself is not available before stage $S$. And that's not a
contradiction. Moreover, if we embrace the cumulative-iterative
conception of set, then we shouldn't even have \emph{expected} to be
able to form the Russell set itself. For that would be the set of all
non-self-membered sets that ``will ever be available''. In short: the
fact that we (provably) can't form the Russell set isn't
\emph{surprising}, given the cumulative-iterative story; it's what we
would \emph{predict}.

%In one sense, then, the cumulative-iterative conception of set yields a response to Russell's Paradox which is rather like the \emph{predicativist}'s response. After all, the predicativist said that the collection of all non-self-membered sets$_n$ is a set$_{n+1}$, and this set$_n$ will not be self-membered. (Indeed, most predicativists will treat it as \emph{ungrammatical} to try to ask whether a set is self-membered.)
%
%But there is an important difference between the cumulative-iterative approach and the predicativist approach: the cumulative-iterative approach treats all of our entities as being \emph{of the same kind}. The predicativist will presumably have an empty set$_0$ (i.e., a set$_0$ with no !!{element}s), and an empty set$_1$ (i.e., a set$_1$ with no !!{element}s), and these will be \emph{different entities}. But on the cumulative-iterative approach, there is just one empty set, $\emptyset$, and it will be available at every stage of the hierarchy. 

%Indeed, the Axiom of Extensionality, stated at the start of this chapter, will hold true of the elements in this hierarchy (so that there can be \emph{only one} ``empty'' set). But I do not intend to use the cumulative-iterative conception to justify Extensionality (see \cite{Boolos1971}). Again: I suggest that we should accept Extensionality, just because we are interested in extensional collections. 

\end{document}