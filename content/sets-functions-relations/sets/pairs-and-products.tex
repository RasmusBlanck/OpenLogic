% Part: sets-functions-relations
% Chapter: sets
% Section: pairs-and-products

\documentclass[../../../include/open-logic-section]{subfiles}

\begin{document}

\olfileid{sfr}{set}{pai}
\olsection{Pairs, Tuples, Cartesian Products}

\begin{explain}
It follows from extensionality that sets have no order to their
elements. So if we want to represent order, we use \emph{ordered
pairs} $\tuple{x, y}$. In an unordered pair $\{x, y\}$, the order does
not matter: $\{x, y\} = \{y, x\}$. In an ordered pair, it does: if $x
\neq y$, then $\tuple{x, y} \neq \tuple{y, x}$.

How should we think about ordered pairs in set theory? Crucially, we
want to preserve the idea that ordered pairs are identical iff they
share the same first element and share the same second element, i.e.:
\[
  \tuple{a, b}= \tuple{c, d}\text{ iff both }a = c \text{ and }b=d.
\]
We can define ordered pairs in set theory using the Wiener-Kuratowski
definition.
\end{explain}

\begin{defn}[Ordered pair]\ollabel{wienerkuratowski}
	$\tuple{a, b} = \{\{a\}, \{a, b\}\}$.
\end{defn}

\begin{prob}
	Using \olref[sfr][set][pai]{wienerkuratowski}, prove that $\tuple{a,
	b}= \tuple{c, d}$ iff both $a = c$ and $b=d$.
\end{prob}

\begin{explain}
Having fixed a definition of an ordered pair, we can use it to define
further sets. For example, sometimes we also want ordered sequences of
more than two objects, e.g., \emph{triples} $\tuple{x, y, z}$,
\emph{quadruples} $\tuple{x, y, z, u}$, and so on.  We can think of
triples as special ordered pairs, where the first element is itself an
ordered pair: $\tuple{x, y, z}$ is $\tuple{\tuple{x, y},z}$. The same
is true for quadruples: $\tuple{x,y,z,u}$ is
$\tuple{\tuple{\tuple{x,y},z},u}$, and so on. In general, we talk of
\emph{ordered $n$-tuples} $\tuple{x_1, \dots, x_n}$.

Certain sets of ordered pairs, or other ordered $n$-tuples, will be useful.
\end{explain}

\begin{defn}[Cartesian product]
Given sets $A$ and $B$, their \emph{Cartesian product} $A \times B$ is
defined by
\[
  A \times B = \Setabs{\tuple{x, y}}{x \in A \text{ and } y \in B}.
\]
\end{defn}

\begin{ex}
If $A = \{0, 1\}$, and $B = \{1, a, b\}$, then their product is
\[
A \times B = \{ \tuple{0, 1}, \tuple{0, a}, \tuple{0, b},
    \tuple{1, 1}, \tuple{1, a}, \tuple{1, b} \}.
\]
\end{ex}

\begin{ex}
If $A$ is a set, the product of $A$ with itself, $A \times A$, is also
written~$A^2$. It is the set of \emph{all} pairs $\tuple{x, y}$ with
$x, y \in A$. The set of all triples $\tuple{x, y, z}$ is $A^3$, and
so on. We can give a recursive definition:
\begin{align*}
  A^1 & = A\\
  A^{k+1} & = A^k \times A
\end{align*}
\end{ex}

\begin{prob}
List all !!{element}s of $\{1, 2, 3\}^3$.
\end{prob}

\begin{prop}\ollabel{cardnmprod}
If $A$ has $n$ !!{element}s and $B$ has $m$ !!{element}s, then $A
\times B$ has $n\cdot m$ elements.
\end{prop}

\begin{proof}
For every !!{element}~$x$ in~$A$, there are $m$ !!{element}s of the
form $\tuple{x, y} \in A \times B$. Let $B_x = \Setabs{\tuple{x, y}}{y
  \in B}$. Since whenever $x_1 \neq x_2$, $\tuple{x_1, y} \neq
\tuple{x_2, y}$, $B_{x_1} \cap B_{x_2} = \emptyset$. But if $A = \{x_1,
\dots, x_n\}$, then $A \times B = B_{x_1} \cup \dots \cup B_{x_n}$, and so has
$n\cdot m$ !!{element}s.

To visualize this, arrange the !!{element}s of~$A \times B$ in a grid:
\[
\begin{array}{rcccc}
  B_{x_1} = & \{\tuple{x_1, y_1} & \tuple{x_1, y_2} & \dots & \tuple{x_1, y_m}\}\\
  B_{x_2} = & \{\tuple{x_2, y_1} & \tuple{x_2, y_2} & \dots & \tuple{x_2, y_m}\}\\
  \vdots & & \vdots\\
  B_{x_n} = & \{\tuple{x_n, y_1} & \tuple{x_n, y_2} & \dots & \tuple{x_n, y_m}\}
\end{array}
\]
Since the $x_i$ are all different, and the $y_j$ are all different, no
two of the pairs in this grid are the same, and there are $n\cdot m$
of them.
\end{proof}

\begin{prob}
Show, by induction on~$k$, that for all $k \ge 1$, if $A$ has $n$
!!{element}s, then $A^k$ has $n^k$ !!{element}s.
\end{prob}

\begin{ex}
If $A$ is a set, a \emph{word} over~$A$ is any sequence of
!!{element}s of~$A$. A sequence can be thought of as an $n$-tuple of
!!{element}s of~$A$. For instance, if $A = \{a, b, c\}$, then the
sequence ``$bac$'' can be thought of as the triple~$\tuple{b, a, c}$.
Words, i.e., sequences of symbols, are of crucial importance in
computer science. By convention, we count !!{element}s of~$A$ as
sequences of length~$1$, and $\emptyset$ as the sequence of length~$0$.
The set of \emph{all} words over~$A$ then is
\[
A^* = \{\emptyset\} \cup A \cup A^2 \cup A^3 \cup \dots
\]
\end{ex}

\end{document}
