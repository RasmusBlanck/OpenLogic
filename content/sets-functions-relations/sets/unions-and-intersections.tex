% Part: sets-functions-relations
% Chapter: sets
% Section: unions-and-intersections

\documentclass[../../../include/open-logic-section]{subfiles}

\begin{document}

\olfileid{sfr}{set}{uni}
\olsection{Unions and Intersections}

\begin{explain}
In \olref[sfr][set][bas]{sec}, we introduced definitions of sets by
abstraction, i.e., definitions of the form $\Setabs{x}{\phi(x)}$.
Here, we invoke some property~$\phi$, and this property can mention
sets we've already defined. So for instance, if $A$ and~$B$ are sets,
the set $\Setabs{x}{x \in A \lor x \in B}$ consists of all those
objects which are !!{element}s of either $A$ or~$B$, i.e., it's the
set that combines the !!{element}s of $A$ and~$B$. We can visualize
this as in \olref{fig:union}, where the highlighted area indicates the
!!{element}s of the two sets $A$ and~$B$ together.

\begin{figure}
  \olasset{assets/diagrams/union.tikz}
  \caption{The union $A \cup B$ of two sets is set of !!{element}s of
   $A$ together with those of~$B$.}
  \ollabel{fig:union} 
\end{figure}

This operation on sets---combining them---is very useful and common,
and so we give it a formal name and a symbol. 
\end{explain}

\begin{defn}[Union]
The \emph{union} of two sets $A$ and $B$, written $A \cup B$, is the
set of all things which are !!{element}s of $A$, $B$, or both.
\[
A \cup B = \Setabs{x}{x \in A \lor x \in B}
\]
\end{defn}

\begin{ex}
Since the multiplicity of !!{element}s doesn't matter, the union of two
sets which have !!a{element} in common contains that !!{element} only once,
e.g., $\{ a, b, c\} \cup \{ a, 0, 1\} = \{a, b, c, 0, 1\}$.

The union of a set and one of its subsets is just the bigger set: $\{a,
b, c \} \cup \{a \} = \{a, b, c\}$.

The union of a set with the empty set is identical to the set: $\{a,
b, c \} \cup \emptyset = \{a, b, c \}$.
\end{ex}

\begin{prob}
Prove that if $A \subseteq B$, then $A \cup B = B$.
\end{prob}

\begin{explain}
We can also consider a ``dual'' operation to union. This is the
operation that forms the set of all !!{element}s that are !!{element}s
of~$A$ and are also !!{element}s of~$B$. This operation is called 
\emph{intersection}, and can be depicted as in \olref{fig:intersection}.
\begin{figure}
  \olasset{assets/diagrams/intersection.tikz}
  \caption{The intersection $A \cap B$ of two sets is the set of
    !!{element}s they have in common.}
  \ollabel{fig:intersection}
\end{figure}
\end{explain}

\begin{defn}[Intersection]
The \emph{intersection} of two sets $A$ and $B$, written $A \cap B$, is
the set of all things which are !!{element}s of both $A$ and~$B$.
\[
A \cap B = \Setabs{x}{x \in A \land x \in B}
\]
Two sets are called \emph{disjoint} if their intersection is
empty. This means they have no !!{element}s in common.
\end{defn}

\begin{ex}
If two sets have no !!{element}s in common, their intersection is empty:
$\{ a, b, c\} \cap \{ 0, 1\} = \emptyset$.

If two sets do have !!{element}s in common, their intersection is the set of
all those: $\{a, b, c \} \cap \{a, b, d \} = \{a, b\}$.

The intersection of a set with one of its subsets is just the smaller
set: $\{a, b, c\} \cap \{a, b\} = \{a, b\}$.

The intersection of any set with the empty set is empty: $\{a, b, c \}
\cap \emptyset = \emptyset$.
\end{ex}

\begin{prob}
Prove rigorously that if $A \subseteq B$, then $A \cap B = A$.
\end{prob}

\begin{explain}
We can also form the union or intersection of more than two
sets. An elegant way of dealing with this in general is the
following: suppose you collect all the sets you want to form the union
(or intersection) of into a single set. Then we can define the union
of all our original sets as the set of all objects which belong to at
least one !!{element} of the set, and the intersection as the set of
all objects which belong to every !!{element} of the set.
\end{explain}

\begin{defn}
If $A$ is a set of sets, then $\bigcup A$ is the set of !!{element}s of
!!{element}s of~$A$:
\begin{align*}
\bigcup A & = \Setabs{x}{x \text{ belongs to !!a{element} of } A},
\text{ i.e.,}\\
& = \Setabs{x}{\text{there is a } B \in A
  \text{ so that } x \in B}
\end{align*}
\end{defn}

\begin{defn}
If $A$ is a set of sets, then $\bigcap A$ is the set of objects which
all elements of~$A$ have in common:
\begin{align*}
\bigcap A & = \Setabs{x}{x \text{ belongs to every !!{element} of } A},
\text{ i.e.,}\\
 & = \Setabs{x}{\text{for all } B \in A, x \in B}
\end{align*}
\end{defn}

\begin{ex}
Suppose $A = \{ \{ a, b \}, \{ a, d, e \}, \{ a, d \} \}$.
Then $\bigcup A = \{ a, b, d, e \}$ and $\bigcap A = \{ a \}$.
\end{ex}
\begin{prob}
	Show that if $A$ is a set and $A \in B$, then $A \subseteq \bigcup B$.
\end{prob}

We could also do the same for a sequence of sets $A_1$, $A_2$, \dots
\begin{align*}
\bigcup_i A_i & = \Setabs{x}{x \text{ belongs to one of the } A_i}\\
\bigcap_i A_i & = \Setabs{x}{x \text{ belongs to every } A_i}.
\end{align*}

When we have an \emph{index} of sets, i.e., some set $I$ such that we
are considering $A_i$ for each $i \in I$, we may also use these
abbreviations:
\begin{align*}
	\bigcup_{i \in I} A_i & = \bigcup \Setabs{A_i }{i \in I}\\
	\bigcap_{i \in I} A_i & = \bigcap\Setabs{A_i}{i \in I}
\end{align*}

Finally, we may want to think about the set of all !!{element}s in~$A$
which are not in~$B$. We can depict this as in \olref{difference}.

\begin{figure}
  \olasset{assets/diagrams/difference.tikz}
  \caption{The difference $A \setminus B$ of two sets is the set of
    those !!{element}s of~$A$ which are not also !!{element}s of~$B$.}
  \ollabel{difference}
\end{figure}

\begin{defn}[Difference]
The \emph{set difference}~$A \setminus B$ is the set of all !!{element}s of
$A$ which are not also !!{element}s of~$B$, i.e.,
\[
A\setminus B = \Setabs{x}{x\in A \text{ and } x \notin B}.
\]
\end{defn}

\begin{prob}
	Prove that if $A \subsetneq B$, then $B \setminus A \neq \emptyset$.
\end{prob}

\end{document}
