% Part:sets-functions-relations
% Chapter: size-of-sets
% Section: equinumerous-sets

\documentclass[../../../include/open-logic-section]{subfiles}

\begin{document}

\olfileid{sfr}{siz}{equ}

\olsection{Equinumerosity}

We have an intuitive notion of ``size'' of sets, which works fine for
finite sets. But what about infinite sets? If we want to come up with
a formal way of comparing the sizes of two sets of \emph{any} size, it
is a good idea to start by defining when sets are the same size. Here
is Frege:
\begin{quote}
  If a waiter wants to be sure that he has laid exactly as many knives
  as plates on the table, he does not need to count either of them, if
  he simply lays a knife to the right of each plate, so that every
  knife on the table lies to the right of some plate. The plates and
  knives are thus uniquely correlated to each other, and indeed
  through that same spatial relationship. \citep[\S70]{Frege1884}
\end{quote}
The insight of this passage can be brought out through a formal
definition:

\begin{defn}\ollabel{comparisondef}
  $A$ is \emph{equinumerous} with $B$, written $\cardeq{A}{B}$, iff
  there is !!a{bijection} $f \colon A \to B$. 
\end{defn}

\begin{prop}\ollabel{equinumerosityisequi}
Equinumerosity is an equivalence relation.
\end{prop}

\begin{proof} 
We must show that equinumerosity is reflexive, symmetric, and
transitive. Let $A, B$, and $C$ be sets.

\emph{Reflexivity.} The identity map $\Id{A} \colon A \to A$, where
$\Id{A} (x) = x$ for all $x \in A$, is !!a{bijection}. So
$\cardeq{A}{A}$.

\emph{Symmetry.} Suppose $\cardeq{A}{B}$, i.e., there is
!!a{bijection} $f\colon A \to B$. Since $f$ is !!{bijective}, its
inverse $f^{-1}$ exists and is also !!{bijective}. Hence,
$f^{-1}\colon B \to A$ is !!a{bijection}, so $\cardeq{B}{A}$.

\emph{Transitivity.} Suppose that $\cardeq{A}{B}$ and $\cardeq{B}{C}$,
i.e., there are !!{bijection}s $f\colon A \to B$ and $g\colon B \to
C$. Then the composition $\comp{f}{g}\colon A \to C$ is !!{bijective},
so that $\cardeq{A}{C}$.
\end{proof}

\begin{prop}
If $\cardeq{A}{B}$, then $A$ is !!{enumerable} if
and only if $B$ is.
\end{prop}

\begin{editorial}
The following proof uses \olref[enm]{defn:enumerable} if
\olref[enm]{sec} is included and \olref[enm-alt]{defn:enumerable}
otherwise.
\end{editorial}

\begin{proof}
Suppose $\cardeq{A}{B}$, so there is some !!{bijection} $f \colon A
\to B$, and suppose that $A$ is !!{enumerable}.
\oliflabeldef{sfr:siz:enm:defn:enumerable}{
  Then either $A = \emptyset$ or there is !!a{surjective} function
  $g\colon \PosInt \to A$. If $A = \emptyset$, then $B = \emptyset$
  also (otherwise there would be !!a{element}~$y \in B$ but no $x \in
  A$ with $g(x) = y$). If, on the other hand, $g\colon \PosInt \to A$
  is !!{surjective}, then $\comp{f}{g} \colon \PosInt \to B$ is
  !!{surjective}. To see this, let $y \in B$. Since $g$ is
  !!{surjective}, there is an $x \in A$ such that $g(x) = y$. Since
  $f$ is !!{surjective}, there is an $n \in \PosInt$ such that $f(n) =
  x$. Hence,
\[
(\comp{f}{g})(n) = g(f(n)) = g(x) = y
\]
and thus $\comp{f}{g}$ is !!{surjective}. We have that $\comp{f}{g}$
is an enumeration of~$B$, and so $B$~is !!{enumerable}.}
{
  Then either $A  = \emptyset$ or there is !!a{bijection}~$g$ whose
  range is $A$ and whose domain is either $\Nat$ or an initial
  sequence of natural numbers. If $A = \emptyset$, then $B =
  \emptyset$ also (otherwise there would be some~$y \in B$ with no $x
  \in A$ such that $g(x) = y$). So suppose we have our
  !!{bijection}~$g$. Then $\comp{g}{f}$ is !!a{bijection} with
  range~$B$ and domain the same as that of~$g$ (i.e., either $\Nat$ or
  an initial segment of it), so that $B$ is !!{enumerable}.}

If $B$ is !!{enumerable}, we obtain that $A$ is !!{enumerable} by
repeating the argument with the !!{bijection} $f^{-1}\colon B \to A$
instead of~$f$. 
\end{proof}

\begin{prob}
Show that if $\cardeq{A}{C}$ and $\cardeq{B}{D}$, and $A \cap B =
C \cap D = \emptyset$, then $\cardeq{A \cup B}{C \cup D}$.
\end{prob}

\begin{prob}
Show that if $A$ is infinite and !!{enumerable}, then
$\cardeq{A}{\Nat}$.
\end{prob}

\end{document}
