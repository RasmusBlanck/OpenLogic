% Part:sets-functions-relations
% Chapter: size-of-sets
% Section: non-enumerability-alt

\documentclass[../../../include/open-logic-section]{subfiles}

\begin{document}

\olfileid{sfr}{siz}{nen-alt}

\olsection{\printtoken{S}{nonenumerable} Sets}

\begin{editorial}
  This section proves the non-enumerability of $\Bin^\omega$ and
  $\Pow{\Nat}$ using the definitions in \olref[enm-alt]{sec}, i.e.,
  requiring a bijection with~$\Nat$ instead of a surjection from
  $\PosInt$.
\end{editorial}

\begin{explain}
The set $\Nat$ of natural numbers is infinite. It is also trivially
!!{enumerable}. But the remarkable fact is that there are
\emph{!!{nonenumerable}} sets, i.e., sets which are not !!{enumerable}
(see \olref[sfr][siz][enm-alt]{defn:enumerable}).

This might be surprising. After all, to say that $A$ is
!!{nonenumerable} is to say that there is \emph{no} !!{bijection} $f
\colon \Nat \to A$; that is, no function mapping the infinitely many
!!{element}s of~$\Nat$ to~$A$ exhausts all of~$A$.  So if $A$ is
!!{nonenumerable}, there are ``more'' !!{element}s of~$A$ than there
are natural numbers.

To prove that a set is !!{nonenumerable}, you have to show that no
appropriate !!{bijection} can exist. The best way to do this is to
show that every attempt to enumerate !!{element}s of~$A$ must leave at
least one !!{element} out; this shows that no function $f\colon \Nat
\to A$ is !!{surjective}. And a general strategy for establishing this
is to use Cantor's \emph{diagonal method}. Given a list of
!!{element}s of $A$, say, $x_1$, $x_2$, \dots, we construct another
!!{element} of~$A$ which, by its construction, cannot possibly be on
that list.

But all of this is best understood by example. So, our first example
is the set~$\Bin^\omega$ of all infinite strings of $0$'s and $1$'s.
(The `$\Bin$' stands for binary, and we can just think of it as the
two-element set
$\{0,1\}$.)\oliflabeldef{sfr:card-arithmetic:card-opps:sec}{\footnote{More
accurately, we should stipulate that $\Bin^\omega$ is the set of all
$\omega$-sequences of $0$'s and $1$s, i.e., the set
$\funfromto{\omega}{\{0,1\}}$. But the meaning of this will only
become clear in \olref[sfr][card-arithmetic][card-opps]{sec}.}{} This
slightly loose formulation should not cause any confusions
for now, however.}
\end{explain}

\begin{thm}
\ollabel{thm:nonenum-bin-omega}
$\Bin^\omega$~is !!{nonenumerable}.
\end{thm}

\begin{proof}
Consider any enumeration of a subset of $\Bin^\omega$. So we have some
list $s_{0}$, $s_{1}$, $s_{2}$, \dots{} where every $s_n$ is an
infinite string of $0$'s and~$1$'s. Let $s_n(m)$ be the $n$th digit of
the $m$th string in this list. So we can now think of our list as an
array, where $s_n(m)$ is placed at the $n$th row and $m$th column:
\[
\begin{array}{c|c|c|c|c|c}
& 0 & 1 & 2 & 3 & \dots \\\hline
0 & \mathbf{s_{0}(0)} & s_{0}(1) & s_{0}(2) & s_0(3) & \dots \\\hline
1 & s_{1}(0)& \mathbf{s_{1}(1)} & s_1(2) & s_1(3) & \dots \\\hline
2 & s_{2}(0)& s_{2}(1) & \mathbf{s_2(2)} & s_2(3) & \dots \\\hline
3 & s_{3}(0)& s_{3}(1) & s_3(2) & \mathbf{s_3(3)} & \dots \\\hline
\vdots & \vdots & \vdots & \vdots & \vdots & \mathbf{\ddots}
\end{array}
\]
We will now construct an infinite string, $d$, of $0$'s and $1$'s
which is not on this list.  We will do this by specifying each of its
entries, i.e., we specify $d(n)$ for all $n \in \Nat$.  Intuitively,
we do this by reading down the diagonal of the array above (hence the
name ``diagonal method'') and then changing every $1$ to a $0$ and
every $1$ to a~$0$. More abstractly, we define $d(n)$ to be $0$ or $1$
according to whether the $n$-th !!{element} of the diagonal, $s_n(n)$,
is $1$ or $0$, that is:
\[
d(n) =
\begin{cases}
1 & \text{if $s_{n}(n) = 0$}\\
0 & \text{if $s_{n}(n) = 1$}
\end{cases}
\]
Clearly $d \in \Bin^\omega$, since it is an infinite string of $0$'s
and $1$'s. But we have constructed $d$ so that $d(n) \neq s_n(n)$ for
any $n \in \Nat$. That is, $d$ differs from $s_n$ in its $n$th entry.
So $d \neq s_n$ for any $n\in \Nat$. So $d$ cannot be on the list
$s_0$, $s_1$, $s_2$,
\dots

We have shown, given an arbitrary enumeration of some subset of
$\Bin^\omega$, that it will omit some !!{element} of $\Bin^\omega$. So
there is no enumeration of the set $\Bin^\omega$, i.e., $\Bin^\omega$
is !!{nonenumerable}.
\end{proof}

\begin{explain}
This proof method is called ``diagonalization'' because it uses the
diagonal of the array to define~$d$. However, diagonalization need
not involve the presence of an array. Indeed, we can show that some set is 
!!{nonenumerable} by using a similar idea, even when no array and no
actual diagonal is involved. The following result illustrates how.
\end{explain}

\begin{thm}
\ollabel{thm:nonenum-pownat}
$\Pow{\Nat}$ is not !!{enumerable}.
\end{thm}

\begin{proof}
We proceed in the same way, by showing that every list of subsets
of~$\Nat$ omits some subset of $\Nat$. So, suppose that we have some
list $N_0, N_1, N_2, \ldots$ of subsets of $\Nat$. We define a set $D$
as follows: $n \in D$ iff $n \notin N_{n}$:
\[
D = \Setabs{n \in \Nat}{n \notin N_n}
\]
Clearly $D\subseteq \Nat$. But $D$ cannot be on the list. After all,
by construction $n \in D$ iff $n\notin N_n$, so that $D \neq N_n$ for
any $n \in \Nat$. 
\end{proof}

\begin{explain}
The preceding proof did not mention a diagonal. Still, you can think
of it as involving a diagonal if you picture it this way: Imagine the
sets $N_0$, $N_1$, \dots, written in an array, where we write $N_n$ on
the $n$th row by writing $m$ in the $m$th column iff if $m \in N_n$.
For example, say the first four sets on that list are
$\{0,1,2,\dots\}$, $\{1, 3, 5, \dots\}$, $\{0,1,4\}$, and
$\{2,3,4,\dots\}$; then our array would begin with
\[
\begin{array}{r@{}rrrrrrr}
  N_0 = \{ & \mathbf{0}, & 1, & 2, & & & & \dots\}\\
  N_1 = \{ &  & \mathbf{1}, &  & 3, &  & 5, & \dots\}\\
  N_2 = \{ & 0, & 1, &  &  & 4\phantom{,} &  & \}\\
  N_3 = \{ &  &  & 2, & \mathbf{3}, & 4, & & \dots\}\\
  &\vdots & & & & & & \ddots\phantom{\}}
  \end{array}
\]
Then $D$ is the set obtained by going down the diagonal, placing $n
\in D$ iff $n$ is \emph{not} on the diagonal. So in the above case, we
would leave out $0$ and $1$, we would include~$2$, we would leave
out~$3$, etc.
\end{explain}

\begin{prob}
Show that the set of all functions $f \colon \Nat \to \Nat$ is
!!{nonenumerable} by an explicit diagonal argument. That is, show that
if $f_1$, $f_2$, \dots, is a list of functions and each $f_i\colon
\Nat \to \Nat$, then there is some $g \colon \Nat \to
\Nat$ not on this list.
\end{prob}

\end{document}