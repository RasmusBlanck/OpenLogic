% Part:sets-functions-relations
% Chapter: sets
% Section: schroder-bernstein

\documentclass[../../../include/open-logic-section]{subfiles}

\begin{document}

\olfileid{sfr}{siz}{sb}

\olsection{The Notion of Size, and Schr\"oder-Bernstein}

\begin{explain}
Here is an intuitive thought: if $A$ is no larger than $B$ and $B$ is
no larger than $A$, then $A$ and $B$ are equinumerous. To be honest,
if this thought were \emph{wrong}, then we could scarcely justify the
thought that our defined notion of equinumerosity has anything to do
with comparisons of ``sizes'' between sets!{} Fortunately, though,
the intuitive thought is correct. This is justified by the
Schr\"oder-Bernstein Theorem.
\end{explain}

\begin{thm}[Schr\"oder-Bernstein]
	\ollabel{thm:schroder-bernstein}
	If $\cardle{A}{B}$ and $\cardle{B}{A}$,
	then $\cardeq{A}{B}$.
\end{thm}

\begin{explain}
In other words, if there is !!a{injection} from $A$ to~$B$, and
!!a{injection} from $B$ to~$A$, then there is !!a{bijection} from $A$
to~$B$. 

This result, however, is really rather \emph{difficult} to prove.
Indeed, although Cantor stated the result, others proved
it.\footnote{For more on the history, see e.g.,
\citet[pp.~165--6]{Potter2004}.}
\oliflabeldef{sfr:cardinals:card-sb:sec}{We will only be in
a position to \emph{prove} Schr\"oder-Bernstein in
\olref[sfr][cardinals][card-sb]{sec}.}{}% 
For now, you can (and must)
take it on trust. 

Fortunately, Schr\"oder-Bernstein is \emph{correct}, and it
vindicates our thinking of the relations we defined, i.e.,
$\cardeq{A}{B}$ and $\cardle{A}{B}$, as having something to do with
``size''. Moreover, Schr\"oder-Bernstein is very \emph{useful}. It
can be difficult to think of !!a{bijection} between two equinumerous
sets. The Schr\"oder-Bernstein Theorem allows us to break the comparison
down into cases so we only have to think of !!a{injection} from the
first to the second, and vice-versa.
\end{explain}
\end{document}