% Part: turing-machines
% Chapter: machines-computations
% Section: combining-machines

\documentclass[../../../include/open-logic-section]{subfiles}

\begin{document}

\olfileid{tur}{mac}{cmb}
\olsection{Combining Turing Machines}

\begin{explain}
The examples of Turing machines we have seen so far have been fairly
simple in nature. But in fact, any problem that can be solved with any
modern programming language can also be solved with Turing machines.
To build more complex Turing machines, it is important to convince
ourselves that we can combine them, so we can build machines to solve
more complex problems by breaking the procedure into simpler parts.
If we can find a natural way to break a complex problem down into
constituent parts, we can tackle the problem in several stages,
creating several simple Turing machines and combining them into one
machine that can solve the problem. This point is especially important
when tackling the Halting Problem in the next section.
\end{explain}

\begin{ex}
\emph{Combining Machines:} Design a machine that computes the function
$f(m,n) = 2(m+n)$.

In order to build this machine, we can combine two machines we are already
familiar with: the addition machine, and the doubler. We begin by drawing 
a state diagram for the addition machine.
\[
\begin{tikzpicture}[->,>=stealth',shorten >=1pt,auto,node distance=2.8cm,
                    semithick]
  \tikzstyle{every state}=[fill=none,draw=black,text=black]

  \node[initial,state] (A)              {$q_0$};
  \node[state]         (B) [right of=A] {$q_1$};
  \node[state]         (C) [right of=B] {$q_2$};

  \path (A) edge node {\TMtrans{\TMblank}{\TMstroke}{\TMright}} (B)
            edge [loop above] node {\TMtrans{\TMstroke}{\TMstroke}{\TMright}} (B)
        (B) edge [loop above] node {\TMtrans{\TMstroke}{\TMstroke}{\TMright}} (B)
            edge node {\TMtrans{\TMblank}{\TMblank}{\TMleft}} (C)
        (C) edge [loop above] node {\TMtrans{\TMstroke}{\TMblank}{\TMstay}} (C);
\end{tikzpicture}
\]
Instead of halting at state $q_2$, we want to continue operation in
order to double the output. Recall that the doubler machine erases
the first stroke in the input and writes two strokes in a separate
output. Let's add an instruction to make sure the tape head is reading 
the first stroke of the output of the addition machine.
\[
\begin{tikzpicture}[->,>=stealth',shorten >=1pt,auto,node distance=2.8cm,
                    semithick]
  \tikzstyle{every state}=[fill=none,draw=black,text=black]

  \node[initial,state] (A)              {$q_0$};
  \node[state]         (B) [right of=A] {$q_1$};
  \node[state]         (C) [right of=B] {$q_2$};
  \node[state]         (D) [below left of=C] {$q_3$};
  \node[state]         (E) [below left of=D] {$q_4$};

  \path (A) edge node {\TMtrans{\TMblank}{\TMstroke}{\TMright}} (B)
            edge [loop above] node {\TMtrans{\TMstroke}{\TMstroke}{\TMright}} (B)
        (B) edge [loop above] node {\TMtrans{\TMstroke}{\TMstroke}{\TMright}} (B)
            edge node {\TMtrans{\TMblank}{\TMblank}{\TMleft}} (C)
        (C) edge node {\TMtrans{\TMstroke}{\TMblank}{\TMleft}} (D)
        (D) edge [loop left] node {\TMtrans{\TMstroke}{\TMstroke}{\TMleft}} (D)
            edge node[left, xshift=-2mm] {\TMtrans{\TMendtape}{\TMendtape}{\TMright}} (E);
\end{tikzpicture}
\]
It is now easy to double the input---all we have to do is connect the
doubler machine onto state $q_4$. This requires renaming the states
of the doubler machine so that they start at $q_4$ instead of
$q_0$---this way we don't end up with two starting states. The final
diagram should look as in \olref{fig:combined}.
\begin{figure}
\[
\begin{tikzpicture}[->,>=stealth',shorten >=1pt,auto,node distance=2.8cm,
                    semithick]
  \tikzstyle{every state}=[fill=none,draw=black,text=black]
  \node[initial,state] (A)              {$q_0$};
  \node[state]         (B) [right of=A] {$q_1$};
  \node[state]         (C) [right of=B] {$q_2$};
  \node[state]         (D) [below left of=C] {$q_3$};
  \node[state]         (E) [below left of=D] {$q_4$};
  \node[state]         (2) [right of=E] {$q_5$};
  \node[state]         (3) [right of=2] {$q_6$};
  \node[state]         (4) [below of=3] {$q_7$};
  \node[state]         (5) [left of=4]  {$q_8$};
  \node[state]         (6) [left of=5]  {$q_9$};

  \path (A) edge node {\TMtrans{\TMblank}{\TMstroke}{\TMright}} (B)
            edge [loop above] node {\TMtrans{\TMstroke}{\TMstroke}{\TMright}} (B)
        (B) edge [loop above] node {\TMtrans{\TMstroke}{\TMstroke}{\TMright}} (B)
            edge node {\TMtrans{\TMblank}{\TMblank}{\TMleft}} (C)
        (C) edge node {\TMtrans{\TMstroke}{\TMblank}{\TMleft}} (D)
        (D) edge [loop left] node {\TMtrans{\TMstroke}{\TMstroke}{\TMleft}} (D)
            edge node[left, xshift=-2mm] {\TMtrans{\TMendtape}{\TMendtape}{\TMright}} (E)
    (E) edge node {\TMtrans{\TMstroke}{\TMblank}{\TMright}} (2)
    (2) edge [loop above] node {\TMtrans{\TMstroke}{\TMstroke}{\TMright}} (2)
      edge node {\TMtrans{\TMblank}{\TMblank}{\TMright}} (3)
    (3) edge [loop above] node {\TMtrans{\TMstroke}{\TMstroke}{\TMright}} (3)
        edge node {\TMtrans{\TMblank}{\TMstroke}{\TMright}} (4)
    (4) edge [loop below] node {\TMtrans{\TMblank}{\TMstroke}{\TMleft}} (4)
        edge node {\TMtrans{\TMstroke}{\TMstroke}{\TMleft}} (5)
    (5) edge [loop below]  node {\TMtrans{\TMstroke}{\TMstroke}{\TMleft}} (5)
        edge              node {\TMtrans{\TMblank}{\TMblank}{\TMleft}} (6)
    (6) edge [loop below] node {\TMtrans{\TMstroke}{\TMstroke}{\TMleft}} (6)
        edge              node {\TMtrans{\TMblank}{\TMblank}{\TMright}} (E);
\end{tikzpicture}
\]
\caption{Combining adder and doubler machines}
\ollabel{fig:combined}
\end{figure}
\end{ex}

\end{document}
