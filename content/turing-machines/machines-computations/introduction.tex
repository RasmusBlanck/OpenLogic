% Part: turing-machines 
% Chapter: machines-computations 
% Section: introduction

\documentclass[../../../include/open-logic-section]{subfiles}

\begin{document}

\olfileid{tur}{mac}{int} 
\olsection{Introduction}

What does it mean for a function, say, from $\Nat$ to $\Nat$ to be
\emph{computable}? Among the first answers, and the most well known
one, is that a function is computable if it can be computed by a
Turing machine. This notion was set out by Alan Turing in 1936.
Turing machines are an example of \emph{a model of computation}---they
are a mathematically precise way of defining the idea of a
``computational procedure.''  What exactly that means is debated, but
it is widely agreed that Turing machines are one way of specifying
computational procedures.  Even though the term ``Turing machine''
evokes the image of a physical machine with moving parts, strictly
speaking a Turing machine is a purely mathematical construct, and as
such it idealizes the idea of a computational procedure.  For
instance, we place no restriction on either the time or memory
requirements of a Turing machine: Turing machines can compute
something even if the computation would require more storage space or
more steps than there are atoms in the universe.

\begin{explain}
It is perhaps best to think of a Turing
machine as a program for a special kind of imaginary mechanism. This
mechanism consists of a \emph{tape} and a \emph{read-write head}. In
our version of Turing machines, the tape is infinite in one direction
(to the right), and it is divided into \emph{squares}, each of which
may contain a symbol from a finite \emph{alphabet}. Such alphabets can
contain any number of different symbols, say, but we will mainly make do
with three: $\TMendtape$, $\TMblank$, and $\TMstroke$. When the
mechanism is started, the tape is empty (i.e., each square contains
the symbol $\TMblank$) except for the leftmost square, which contains
$\TMendtape$, and a finite number of squares which contain the
\emph{input}. At any time, the mechanism is in one of a finite number
of \emph{states}. At the outset, the head scans the leftmost square
and in a specified \emph{initial state}. At each step of the
mechanism's run, the content of the square currently scanned together
with the state the mechanism is in and the Turing machine program
determine what happens next. The Turing machine program is given by a
partial function which takes as input a state~$q$ and a
symbol~$\sigma$ and outputs a triple~$\tuple{q', \sigma',
  D}$. Whenever the mechanism is in state $q$ and reads symbol
$\sigma$, it replaces the symbol on the current square with $\sigma'$,
the head moves left, right, or stays put according to whether $D$ is
$\TMleft$, $\TMright$, or $\TMstay$, and the mechanism goes into
state~$q'$.

For instance, consider the situation in \olref{fig:tm}.
\begin{figure}
  \olasset{\olpath/assets/diagrams/turing-machine.tikz}
  \caption{A Turing machine executing its program.}
  \ollabel{fig:tm}
\end{figure}
The visible part of the tape of the Turing machine contains the
end-of-tape symbol $\TMendtape$ on the leftmost square, followed by
three $1$'s, a $0$, and four more $1$'s.  The head is reading the
third square from the left, which contains a~$1$, and is in
state~$q_1$---we say ``the machine is reading a $1$ in state~$q_1$.''
If the program of the Turing machine returns, for input $\tuple{q_1,
  1}$, the triple $\tuple{q_2, 0, \TMstay}$, the machine would now
replace the $1$ on the third square with a~$0$, leave the read/write
head where it is, and switch to state~$q_2$.  If then the program
returns $\tuple{q_3, 0, \TMright}$ for input $\tuple{q_2, 0}$, the
machine would now overwrite the $0$ with another~$0$ (effectively,
leaving the content of the tape under the read/write head unchanged),
move one square to the right, and enter state~$q_3$. And so on.

We say that the machine \emph{halts} when it encounters some state,
$q_n$, and symbol, $\sigma$ such that there is no instruction for
$\tuple{q_n, \sigma}$, i.e., the transition function for input
$\tuple{q_n,\sigma}$ is undefined. In other words, the machine has no
instruction to carry out, and at that point, it ceases
operation. Halting is sometimes represented by a specific halt
state~$h$.  This will be demonstrated in more detail later on.
\end{explain}

\begin{digress}
The beauty of Turing's paper, ``On computable numbers,'' is that he
presents not only a formal definition, but also an argument that the
definition captures the intuitive notion of computability.
From the definition, it should be clear that any function computable
by a Turing machine is computable in the intuitive sense. Turing
offers three types of argument that the converse is true, i.e., that
any function that we would naturally regard as computable is
computable by such a machine. They are (in Turing's words):
\begin{enumerate}
\item A direct appeal to intuition.
\item A proof of the equivalence of two definitions (in case the new
  definition has a greater intuitive appeal).
\item Giving examples of large classes of numbers which are
  computable.
\end{enumerate}
Our goal is to try to define the notion of computability ``in
principle,'' i.e., without taking into account practical limitations
of time and space. Of course, with the broadest definition of
computability in place, one can then go on to consider computation
with bounded resources; this forms the heart of the subject known as
``computational complexity.''
\end{digress}

\begin{history}
Alan Turing invented Turing machines in 1936. While his interest at
the time was the decidability of first-order logic, the paper has been
described as a definitive paper on the foundations of computer
design. In the paper, Turing focuses on computable real numbers, i.e.,
real numbers whose decimal expansions are computable; but he notes
that it is not hard to adapt his notions to computable functions on
the natural numbers, and so on.  Notice that this was a full five
years before the first working general purpose computer was built in
1941 (by the German Konrad Zuse in his parent's living room), seven
years before Turing and his colleagues at Bletchley Park built the
code-breaking Colossus (1943), nine years before the American ENIAC
(1945), twelve years before the first British general purpose
computer---the Manchester Small-Scale Experimental Machine---was built in
Manchester (1948), and thirteen years before the Americans first
tested the BINAC (1949). The Manchester SSEM has the distinction of
being the first stored-program computer---previous machines had to be
rewired by hand for each new task.
\end{history}

\end{document}
