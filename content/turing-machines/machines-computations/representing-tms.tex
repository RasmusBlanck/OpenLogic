% Part: turing-machines
% Chapter: machines-computations
% Section: representing-tms

\documentclass[../../../include/open-logic-section]{subfiles}

\begin{document}

\olfileid{tms}{tms}{rep}
\olsection{Representing Turing Machines}

\begin{explain}
Turing machines can be represented visually by \emph{state diagrams}. 
The diagrams are composed of
state cells connected by arrows. Unsurprisingly, each state cell represents
a state of the machine. Each arrow represents an instruction that can be
carried out from that state, with the specifics of the instruction written above
or below the appropriate arrow. Consider the following machine, which has 
only two internal states, $q_0$ and $q_1$, and one instruction:
\[
\begin{tikzpicture}[->,>=stealth',shorten >=1pt,auto,node distance=2.8cm,
                    semithick]
  \tikzstyle{every state}=[fill=none,draw=black,text=black]

  \node[initial,state]   (A)              {$q_0$};
  \node[state]   (B) [right of=A] {$q_1$};

  \path (A) edge  node {\TMtrans{\TMblank}{\TMstroke}{\TMright}} (B);
\end{tikzpicture}
\]
Recall that the Turing machine has a read/write head and a tape with
the input written on it. The instruction can be read as \emph{if
reading a~$\TMblank$ in state $q_0$, write a~$\TMstroke$, move right,
and move to state $q_1$}. This is equivalent to the transition
function mapping $\tuple{q_0, \TMblank}$ to $\tuple{q_1, \TMstroke,
\TMright}$.
\end{explain}

\begin{ex}
\emph{Even Machine}: The following Turing machine halts if, and only
if, there are an even number of $\TMstroke$'s on the tape (under the
assumption that all $\TMstroke$'s come before the first $\TMblank$ on
the tape).
\[
\begin{tikzpicture}[->,>=stealth',shorten >=1pt,auto,node distance=2.8cm,
                    semithick]
  \tikzstyle{every state}=[fill=none,draw=black,text=black]

  \node[initial,state]         (A)              {$q_0$};
  \node[state]         (B) [right of=A] {$q_1$};

  \path (A) edge [bend left] node {\TMtrans{\TMstroke}{\TMstroke}{\TMright}} (B)
        (B) edge [loop above] node {\TMtrans{\TMblank}{\TMblank}{\TMright}} (B)
            edge [bend left] node {\TMtrans{\TMstroke}{\TMstroke}{\TMright}} (A);
\end{tikzpicture}
\]

The state diagram corresponds to the following transition function:
\begin{align*}
\delta(q_0, \TMstroke) & = \tuple{q_1, \TMstroke, \TMright},\\
\delta(q_1, \TMstroke) & = \tuple{q_0, \TMstroke, \TMright},\\
\delta(q_1, \TMblank)  & = \tuple{q_1, \TMblank, \TMright}
\end{align*}
\end{ex}

\begin{explain}
The above machine halts only when the input is an even number of strokes.
Otherwise, the machine (theoretically) continues to operate indefinitely. 
For any machine and input, it is possible to trace through the 
\emph{configurations} of the machine in order to determine the output. 
We will give a formal definition of configurations later. For now,
we can intuitively think of configurations as a series of diagrams showing the 
state of the machine at any point in time during operation.
Configurations show the content of the tape, the state of the machine and the
location of the read/write head.

Let us trace through the configurations of the even machine if it is
started with an input of four $\TMstroke$'s. In this case, we expect that the
machine will halt.  We will then run the machine on an input of three
$\TMstroke$'s, where the machine will run forever.

The machine starts in state~$q_0$, scanning the leftmost~$\TMstroke$.
We can represent the initial state of the machine as follows:
\[
\TMendtape \TMstroke_0 \TMstroke \TMstroke \TMstroke \TMblank \ldots
\]
The above configuration is straightforward. As can be seen, the
machine starts in state one, scanning the leftmost~$\TMstroke$. This
is represented by a subscript of the state name on the
first~$\TMstroke$. The applicable instruction at this point is
$\delta(q_0, \TMstroke) = \tuple{q_1, \TMstroke, \TMright}$, and so
the machine moves right on the tape and changes to state~$q_1$.
\[
\TMendtape \TMstroke \TMstroke_1 \TMstroke \TMstroke \TMblank \ldots
\]
Since the machine is now in state~$q_1$ scanning a~$\TMstroke$, we have to
``follow'' the instruction $\delta(q_1, \TMstroke) = \tuple{q_0,
  \TMstroke, \TMright}$. This results in the configuration
\[
\TMendtape \TMstroke \TMstroke \TMstroke_0 \TMstroke \TMblank \ldots
\]
As the machine continues, the rules are applied again in the same
order, resulting in the following two configurations:
\[
\TMendtape \TMstroke \TMstroke \TMstroke \TMstroke_1 \TMblank \ldots
\]
\[
\TMendtape \TMstroke \TMstroke \TMstroke \TMstroke \TMblank_0 \ldots
\]
The machine is now in state~$q_0$ scanning a~$\TMblank$. Based on the
transition diagram, we can easily see that there is no instruction to be
carried out, and thus the machine has halted. This means that the input
has been accepted.

Suppose next we start the machine with an input of three
$\TMstroke$'s. The first few configurations are similar, as the same
instructions are carried out, with only a small difference of the tape
input:
\[
\TMendtape \TMstroke_0 \TMstroke \TMstroke \TMblank \ldots
\]
\[
\TMendtape \TMstroke \TMstroke_1 \TMstroke \TMblank \ldots
\]
\[
\TMendtape \TMstroke \TMstroke \TMstroke_0 \TMblank \ldots
\]
\[
\TMendtape \TMstroke \TMstroke \TMstroke \TMblank_1 \ldots
\]
The machine has now traversed past all the $\TMstroke$'s, and is
reading a~$\TMblank$ in state~$q_1$. As shown in the diagram, there is
an instruction of the form $\delta(q_1, \TMblank) =\tuple{q_1,
\TMblank, \TMright}$. Since the tape is filled with $\TMblank$
indefinitely to the right, the machine will continue to execute this
instruction \emph{forever}, staying in state~$q_1$ and moving ever
further to the right. The machine will never halt, and does not accept
the input.
\end{explain}

\begin{explain}
It is important to note that not all machines will halt. If halting
means that the machine runs out of instructions to execute, then we
can create a machine that never halts simply by ensuring that there is
an outgoing arrow for each symbol at each state. The even machine can
be modified to run indefinitely by adding an instruction for scanning a
$\TMblank$ at~$q_0$.
\end{explain}

\begin{ex}
\[
\begin{tikzpicture}[->,>=stealth',shorten >=1pt,auto,node distance=2.8cm,
                    semithick]
  \tikzstyle{every state}=[fill=none,draw=black,text=black]

  \node[initial,state] (A)              {$q_0$};
  \node[state]         (B) [right of=A] {$q_1$};

  \path
  (A) edge [bend left] node {\TMtrans{\TMstroke}{\TMstroke}{\TMright}} (B)
      edge [loop above] node {\TMtrans{\TMblank}{\TMblank}{\TMright}} (A)
  (B) edge [loop above] node {\TMtrans{\TMblank}{\TMblank}{\TMright}} (B)
      edge [bend left] node {\TMtrans{\TMstroke}{\TMstroke}{\TMright}} (A);
\end{tikzpicture}
\]
\end{ex}

\begin{explain}
Machine tables are another way of representing Turing
machines. Machine tables have the tape alphabet displayed on the
$x$-axis, and the set of machine states across the $y$-axis. Inside the
table, at the intersection of each state and symbol, is written the
rest of the instruction---the new state, new symbol, and direction of
movement. Machine tables make it easy to determine in what state, and
for what symbol, the machine halts. Whenever there is a gap in the
table is a possible point for the machine to halt. Unlike state
diagrams and instruction sets, where the points at which the machine
halts are not always immediately obvious, any halting points are
quickly identified by finding the gaps in the machine table.
\end{explain}

\begin{ex}
The machine table for the even machine is:
\[
\centering
\begin{tabular}{lllll}
\cline{2-3}
\multicolumn{1}{l|}{}      & \multicolumn{1}{l|}{$\TMblank$}                
& \multicolumn{1}{l|}{$\TMstroke$}                &  &  \\ \cline{1-3}
\multicolumn{1}{|l|}{$q_0$} & \multicolumn{1}{l|}{}                          
& \multicolumn{1}{l|}{\TMtrans{\TMstroke}{q_1}{\TMright}} &  &  \\ \cline{1-3}
\multicolumn{1}{|l|}{$q_1$} & \multicolumn{1}{l|}{\TMtrans{\TMblank}{q_1}{\TMblank}} 
& \multicolumn{1}{l|}{\TMtrans{\TMstroke}{q_0}{\TMright}} &  &  \\ \cline{1-3}
\end{tabular}
\]
As we can see, the machine halts when scanning a blank in state~$q_0$.
\end{ex}

\begin{explain}
So far we have only considered machines that read and accept input.
However, Turing machines have the capacity to both read and write. An
example of such a machine (although there are many, many examples) is
a \emph{doubler}. A doubler, when started with a block of~$n$
$\TMstroke$'s on the tape, outputs a block of~$2n$ $\TMstroke$'s.
\end{explain}

\begin{ex}
  \ollabel{ex:doubler} Before building a doubler machine, it is
  important to come up with a \emph{strategy} for solving the problem.
  Since the machine (as we have formulated it) cannot remember how
  many $\TMstroke$'s it has read, we need to come up with a way to
  keep track of all the $\TMstroke$'s on the tape. One such way is to
  separate the output from the input with a~$\TMblank$. The machine
  can then erase the first $\TMstroke$ from the input, traverse over
  the rest of the input, leave a $\TMblank$, and write two new
  $\TMstroke$'s.  The machine will then go back and find the second
  $\TMstroke$ in the input, and double that one as well. For each one
  $\TMstroke$ of input, it will write two $\TMstroke$'s of output.  By
  erasing the input as the machine goes, we can guarantee that no
  $\TMstroke$ is missed or doubled twice. When the entire input is
  erased, there will be $2n$ $\TMstroke$'s left on the tape. The state
  diagram of the resulting Turing machine is depicted in
  \olref{fig:doubler}.
  \begin{figure}
\[
\begin{tikzpicture}[->,>=stealth',shorten >=1pt,auto,node distance=2.8cm,
                    semithick]
  \tikzstyle{every state}=[fill=none,draw=black,text=black]

  \node[initial,state] (1)              {$q_0$};
  \node[state]         (2) [right of=1] {$q_1$};
  \node[state]         (3) [right of=2] {$q_2$};
  \node[state]         (4) [below of=3] {$q_3$};
  \node[state]         (5) [left of=4]  {$q_4$};
  \node[state]         (6) [left of=5]  {$q_5$};

  \path (1) edge node {\TMtrans{\TMstroke}{\TMblank}{\TMright}} (2)
    (2) edge [loop above] node {\TMtrans{\TMstroke}{\TMstroke}{\TMright}} (2)
      edge node {\TMtrans{\TMblank}{\TMblank}{\TMright}} (3)
    (3) edge [loop above] node {\TMtrans{\TMstroke}{\TMstroke}{\TMright}} (3)
        edge node {\TMtrans{\TMblank}{\TMstroke}{\TMright}} (4)
    (4) edge [loop below] node {\TMtrans{\TMblank}{\TMstroke}{\TMleft}} (4)
        edge node {\TMtrans{\TMstroke}{\TMstroke}{\TMleft}} (5)
    (5) edge [loop below]  node {\TMtrans{\TMstroke}{\TMstroke}{\TMleft}} (5)
        edge              node {\TMtrans{\TMblank}{\TMblank}{\TMleft}} (6)
    (6) edge [loop below] node {\TMtrans{\TMstroke}{\TMstroke}{\TMleft}} (6)
        edge              node {\TMtrans{\TMblank}{\TMblank}{\TMright}} (1);
\end{tikzpicture}
\]
\caption{A doubler machine}
\ollabel{fig:doubler}
\end{figure}
\end{ex}

\begin{prob}
Choose an arbitary input and trace through the configurations of the
doubler machine in \olref[tms][tms][rep]{ex:doubler}.
\end{prob}

\begin{prob}
The double machine in \olref[tms][tms][rep]{ex:doubler} writes its
output to the right of the input.  Come up with a new method for
solving the doubler problem which generates its output immediately to
the right of the end-of-tape marker. Build a machine that executes
your method. Check that your machine works by tracing through the
configurations.
\end{prob}

\begin{prob}
Design a Turing-machine with alphabet $\{\TMendtape,\TMblank, A, B\}$
that accepts, i.e., halts on, any string of $A$'s and $B$'s where the
number of $A$'s is the same as the number of $B$'s \emph{and} all the
$A$'s precede all the $B$'s, and rejects, i.e., does not halt on, any
string where the number of $A$'s is not equal to the number of~$B$'s or
the $A$'s do not precede all the~$B$'s. (E.g., the machine should accept
$AABB$, and $AAABBB$, but reject both $AAB$ and $AABBAABB$.)
\end{prob}

\begin{prob}
Design a Turing-machine with alphabet $\{\TMendtape,\TMblank, A, B\}$
that takes as input any string $\alpha$ of $A$'s and $B$'s and
duplicates them to produce an output of the form $\alpha\alpha$. (E.g.
input $ABBA$ should result in output $ABBAABBA$).
\end{prob}

\begin{prob}
\emph{Alphabetical?:} Design a Turing-machine with alphabet
$\{\TMendtape,\TMblank, A, B\}$ that when given as input a finite
sequence of $A$'s and $B$'s checks to see if all the $A$'s appear to
the left of all the $B$'s or not. The machine should leave the input
string on the tape, and either halt if the string is
``alphabetical'', or loop forever if the string is not.
\end{prob}

\begin{prob}
\emph{Alphabetizer:} Design a Turing-machine with alphabet
$\{\TMendtape,\TMblank, A, B\}$ that takes as input a finite sequence
of $A$'s and $B$'s rearranges them so that all the $A$'s are to the
left of all the~$B$'s. (e.g., the sequence $BABAA$ should become the
sequence $AAABB$, and the sequence $ABBABB$ should become the sequence
$AABBBB$).
\end{prob}

\end{document}
